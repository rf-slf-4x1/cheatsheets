% Подписи колонтитула
\newcommand{\colontitulAutors}{astronom\_v\_cube,~edombek}
\newcommand{\colontitulYear}{2023~}
\newcommand{\colontitulEducationalSubject}{Статистическая радиофизика (ТИК) - основные понятия}
\newcommand{\colontitulTeacher}{Мальцев А. А.}

%Настройки шаблона
\documentclass[10pt,landscape,a4paper]{article}
\usepackage[utf8]{inputenc}
\usepackage[english, russian]{babel}
\usepackage[T1,T2A]{fontenc}  
\usepackage{upgreek} % прямые греческие ради русской традиции
\usepackage{tikz}
\usetikzlibrary{shapes,positioning,arrows,fit,calc,graphs,graphs.standard}
%\usepackage[nosf]{kpfonts}
%\usepackage[t1]{sourcesanspro}
\usepackage{multicol}
\usepackage{wrapfig}
\usepackage[top=6mm,bottom=8mm,left=4mm,right=4mm]{geometry}
\usepackage[framemethod=tikz]{mdframed}
\usepackage{microtype}
\usepackage{pdfpages}
\usepackage{amsthm,amsmath,amscd}   % Математические дополнения от AMS
\usepackage{amsfonts,amssymb}       % Математические дополнения от AMS
\usepackage{mathtools}              % Добавляет окружение multlined
\usepackage{xfrac}                  % Красивые дроби
\usepackage{physics}

\usepackage{fancyhdr} % колонтитулы

%некоторые математические команды
\newcommand{\Div}{\operatorname{div}}
\newcommand{\Grad}{\operatorname{grad}}

\let\bar\overline

\definecolor{myblue}{cmyk}{1,.72,0,.38}

\def\firstcircle{(0,0) circle (1.5cm)}
\def\secondcircle{(0:2cm) circle (1.5cm)}

\colorlet{circle edge}{myblue}
\colorlet{circle area}{myblue!5}

\tikzset{filled/.style={fill=circle area, draw=circle edge, thick},
	outline/.style={draw=circle edge, thick}}

\pgfdeclarelayer{background}
\pgfsetlayers{background,main}

%\everymath\expandafter{\the\everymath \color{myblue}}
\everydisplay\expandafter{\the\everydisplay \color{myblue}}

\renewcommand{\baselinestretch}{.8}
\pagestyle{empty}

\global\mdfdefinestyle{header}{%
	linecolor=gray,linewidth=1pt,%
	leftmargin=0mm,rightmargin=0mm,skipbelow=0mm,skipabove=0mm,
}

\makeatletter % Author: ttps://tex.stackexchange.com/questions/218587/how-to-set-one-header-for-each-page-using-multicols
\renewcommand{\section}{\@startsection{section}{1}{0mm}%
	{.2ex}%
	{.2ex}%x
	{\color{myblue}\sffamily\small\bfseries}}
\renewcommand{\subsection}{\@startsection{subsection}{1}{0mm}%
	{.2ex}%
	{.2ex}%x
	{\sffamily\bfseries}}

\makeatother
\setlength{\parindent}{0pt}

%колонтитулы
\pagestyle{fancy}
\fancyhf{}
\setlength{\headheight}{40pt}
\setlength{\headsep}{4pt}
\renewcommand{\headrulewidth}{1pt}
\fancyhead[L]{\textcopyright~\colontitulAutors}
\fancyhead[C]{Программа минимум по курсу <<\colontitulEducationalSubject>> \colontitulYear г}
\fancyhead[R]{Преподаватель:~\colontitulTeacher}

\begin{document}
	\small
	\begin{multicols*}{2}
		\section{Определение и вероятностное описание случайных процессов}
		\textbf{Определение 1.} Если для каждого значения $t$ зависимая переменная $x$ представляет собой случайную величину, то говорят, что $х=x(t)$ есть случайная функция времени или случайный процесс. Случайный процесс по Слуцкому – это однопараметрическое семейство случайных величин\\
		\textbf{Определение 2.}. Случайный процесс (случайная функция времени) $x(t)$ – это есть некое множество детерминированных функций (реализаций случайного процесса) на котором задана вероятностная мера\\
		В определении случайного процесса мы имеем зависимость от времени и ансамбля - случайный процесс является функцией двух переменных. Поэтому в строгой математической литературе используют обозначения ${х(t,\omega), t\in T, \omega \in \Omega}$, где $\Omega$ - выборочное пространство рассматриваемого эксперимента (или пространство элементарных событий), $Т$ – множество индексов времени, непрерывное или дискретное. Тогда для отдельной реализации можно использовать обозначениу: $x(t,\omega_m)$ - детерминированная функция, реализация соответствует событию $\omega_m$ из выборочного пространства $\Omega$\\
		$x(t_1,\omega)$, $x(t_2,\omega)$ – случайные величины, соответствующие значениям случайного процесса в моменты времени $t_1$ и $t_2$, определенные на выборочном пространстве $\Omega$, значения которых и образуют множество элементарных событий или выборочное пространство $\Omega$. Сам случайный процесс записывают как $x(t,\omega)$

		Зная статистический ансамбль случайного процесса, можно найти вероятностную меру (частоту появления реализаций). Плотность вероятности является положительнозначной функцией и удовлетворяет условию нормировки $\int_{-\infty}^{\infty} W(x,t) \,dx = 1$ для любого $t$. Физический смысл $W(x,t)$, как и физический смысл плотности вероятности случайной величины – это вероятность прохождения случайного процесса $x(t)$ через «дельта-ворота», расположенные в точке $x$ и имеющие ширину $dx$.\\

		Среднее статистическое значение: $<x(t)> = \int_{-\infty}^{\infty} x W(x,t) \,dx$ - усреднение по стат. ансамблю\\
		Средний квадрат случайного процесса: $<x^2(t)> = \int_{-\infty}^{\infty} x^2 W(x,t) \,dx$\\

		Двумерная плотность вероятности: $W(x_1,t_1; x_2,t_2) = P$ ЛЕЕЕЕНЬ стр 8 \\
		Физический смысл двумерной плотности вероятности заключается в том, что она определяет вероятность одновременного прохождения реализаций случайного процесса через пару «ворот», расположенных в точках $x_1$ и $x_2$ и имеющих ширину $dx_1$ и $dx_2$\\

		\textbf{Основные свойства n-мерных плотностей вероятности}\\
		1. Неотрицательность: $W(x_1, t_1; x_2, t_2;\ldots; x_n, t_n) \geqslant 0$\\
		2. Нормировка: $\int_{-\infty}^{\infty} W(x_1, t_1; x_2, t_2;\ldots; x_n, t_n)dx_1 dx_2 \ldots dx_n = 1$\\
		3. Симметрия: $W(x_1, t_1; x_2, t_2;\ldots; x_n, t_n) = W(x_2, t_2; x_1, t_1;\ldots; x_n, t_n)$\\
		4. Согласованность: $W_m(x_1, t_1; x_2, t_2;\ldots; x_m, t_m) =$ЛЕЕЕЕНЬ стр 11
		
	\end{multicols*}
\end{document}

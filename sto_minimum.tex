\documentclass[10pt,landscape,a4paper]{article}
\usepackage[utf8]{inputenc}
\usepackage[russian]{babel}
\usepackage[T2A]{fontenc}
\usepackage{tikz}
\usetikzlibrary{shapes,positioning,arrows,fit,calc,graphs,graphs.standard}
%\usepackage[nosf]{kpfonts}
\usepackage[t1]{sourcesanspro}
\usepackage{multicol}
\usepackage{wrapfig}
\usepackage[top=0mm,bottom=1mm,left=0mm,right=1mm]{geometry}
\usepackage[framemethod=tikz]{mdframed}
\usepackage{microtype}
\usepackage{pdfpages}

\let\bar\overline

\documentclass[10pt,landscape,a4paper]{article}
\usepackage[utf8]{inputenc}
\usepackage[english, russian]{babel}
\usepackage[T1,T2A]{fontenc}  
\usepackage{upgreek} % прямые греческие ради русской традиции
\usepackage{tikz}
\usetikzlibrary{shapes,positioning,arrows,fit,calc,graphs,graphs.standard}
%\usepackage[nosf]{kpfonts}
%\usepackage[t1]{sourcesanspro}
\usepackage{multicol}
\usepackage{wrapfig}
\usepackage[top=6mm,bottom=8mm,left=4mm,right=4mm]{geometry}
\usepackage[framemethod=tikz]{mdframed}
\usepackage{microtype}
\usepackage{pdfpages}
\usepackage{amsthm,amsmath,amscd}   % Математические дополнения от AMS
\usepackage{amsfonts,amssymb}       % Математические дополнения от AMS
\usepackage{mathtools}              % Добавляет окружение multlined
\usepackage{xfrac}                  % Красивые дроби
\usepackage{physics}

\usepackage{fancyhdr} % колонтитулы

%некоторые математические команды
\newcommand{\Div}{\operatorname{div}}
\newcommand{\Grad}{\operatorname{grad}}

\let\bar\overline

\definecolor{myblue}{cmyk}{1,.72,0,.38}

\def\firstcircle{(0,0) circle (1.5cm)}
\def\secondcircle{(0:2cm) circle (1.5cm)}

\colorlet{circle edge}{myblue}
\colorlet{circle area}{myblue!5}

\tikzset{filled/.style={fill=circle area, draw=circle edge, thick},
	outline/.style={draw=circle edge, thick}}

\pgfdeclarelayer{background}
\pgfsetlayers{background,main}

%\everymath\expandafter{\the\everymath \color{myblue}}
\everydisplay\expandafter{\the\everydisplay \color{myblue}}

\renewcommand{\baselinestretch}{.8}
\pagestyle{empty}

\global\mdfdefinestyle{header}{%
	linecolor=gray,linewidth=1pt,%
	leftmargin=0mm,rightmargin=0mm,skipbelow=0mm,skipabove=0mm,
}

\makeatletter % Author: ttps://tex.stackexchange.com/questions/218587/how-to-set-one-header-for-each-page-using-multicols
\renewcommand{\section}{\@startsection{section}{1}{0mm}%
	{.2ex}%
	{.2ex}%x
	{\color{myblue}\sffamily\small\bfseries}}
\renewcommand{\subsection}{\@startsection{subsection}{1}{0mm}%
	{.2ex}%
	{.2ex}%x
	{\sffamily\bfseries}}

\makeatother
\setlength{\parindent}{0pt}

%колонтитулы
\pagestyle{fancy}
\fancyhf{}
\setlength{\headheight}{40pt}
\setlength{\headsep}{4pt}
\renewcommand{\headrulewidth}{1pt}
\fancyhead[L]{\textcopyright~\colontitulAutors}
\fancyhead[C]{Программа минимум по курсу <<\colontitulEducationalSubject>> \colontitulYear г}
\fancyhead[R]{Преподаватель:~\colontitulTeacher}

\begin{document}
	\small
	\begin{multicols*}{5}
		\section{Постулаты Эйнштейна}
		\subsection{Постулат относительности}
		\textbf{Законы природы одинаковы во всех ИСО.} Другими словами, законы природы \textbf{ковариантны} по отношению к преобразованиям координат и времени от одной инерциальной СО к другой. Это значит, что уравнения, описывающие некоторый закон природы и выраженные через координаты и время различных ИСО, имеют один и тот же вид.
		\subsection{Постулат постоянства скорости света}
		\textbf{Скорость света не зависит от движения источника и равнас во всех ИСО и по всем направлениям.}
		
		\section{Каноническая форма уравнений Максвелла в вакууме: 4-потенциал и 4-плотность тока в 4-пространстве}
		\begin{fralign*}
			\bar{x} = \left(x, y, z, ict\right) \\
			\Delta \dfrac{1}{c^2}\dfrac{\partial^2}{\partial t^2}=\sum\limits_{s=1}^4\dfrac{\partial^2}{\partial x_s^2}=\Box  \\
			\bar{A} = \left(A_x, A_y, A_z, i\phi\right) \text{-четырёхпотенциал} \\ 
			\bar{J} = \left(j_x, j_y, j_z, ic\rho\right) \text{-четырёхплотность тока}  \\
			\Box \bar{A} = -\dfrac{4\pi}{c}\bar{j}, div{\bar{A}} = 0, div{\bar{J}} = 0 
		\end{fralign*}
		
		\subsection*{Интервал между мировыми координатами двух событий в ИСО. Инвариантность интервала}
		Не
		\section{Преобразования Лоренца}
		\section{Световой конус и мировые линии в 4-мерном пространстве}
		\section{Относительность одновременности двух событий}
		\section{Собственное время объекта}
		\section{Лоренцево сокращение длины движущегося масштаба}
		\section{Закон сложения скоростей}
		\section{Эффект Допплера}
		\section{Действие и функция Лагранжа свободной материальной частицы в ИСО}
		\section{Импульс и энергия свободной материальной частицы}
		\section{Уравнение движения релятивистской частицы в 3-мерном пространстве}
		\section{4-скорость и 4-импульс свободной материальной частицы}
		\section{Ковариантная форма уравнения движения частицы в ИСО и 4-сила Минковского}
		\section{Тензор электромагнитного поля и ковариантная форма уравнений электродинамики в вакууме}
		\section{Форма и содержание закона преобразования полей}
		\section{Инварианты тензора электромагнитного поля}
		\section{4-вектор плотности силы Лоренца и его связь с тензором электромагнитного поля}
		\section{4-вектор плотности силы Лоренца и его связь с электромагнитным тензором энергии-импульса}
		\section{Закон сохранения энергии в электродинамике}
		\section{Закон сохранения импульса в электродинамике}
		\section{Действие и функция Лагранжа заряженной частицы в заданном электромагнитном поле}
		\section{Импульс заряженной частицы в заданном электромагнитном поле}
		\section{Энергия заряженной частицы в заданном электромагнитном поле}
		\section{Уравнение движения заряженной частицы в заданном электромагнитном поле}
		\section{Поле равномерно движущегося заряда}
		\section{Потенциалы Льенара-Вихерта неравномерно движущегося заряда. Выражение для поля излучения}
		\section{Излучение неравномерно движущегося на малой скорости заряда (формула Лармора)}
		\section{Тормозное излучение заряда}
		\section{Синхротронное (магнитотормозное) излучение заряда}
		\section{Излучение Вавилова-Черенкова}
		\section{Гипотезы теории электромагнитной массы и радиус электрона}
		\section{Сила реакции излучения и уравнение Абрагама-Лоренца}
	\end{multicols*}
\end{document}
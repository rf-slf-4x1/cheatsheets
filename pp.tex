% Подписи колонтитула
\newcommand{\colontitulAutors}{astronom\_v\_cube}
\newcommand{\colontitulYear}{2022 }
\newcommand{\colontitulEducationalSubject}{Основы теории полупроводников}
\newcommand{\colontitulTeacher}{Оболенский С.В.}

%Настройки шаблона
\documentclass[10pt,landscape,a4paper]{article}
\usepackage[utf8]{inputenc}
\usepackage[english, russian]{babel}
\usepackage[T1,T2A]{fontenc}  
\usepackage{upgreek} % прямые греческие ради русской традиции
\usepackage{tikz}
\usetikzlibrary{shapes,positioning,arrows,fit,calc,graphs,graphs.standard}
%\usepackage[nosf]{kpfonts}
%\usepackage[t1]{sourcesanspro}
\usepackage{multicol}
\usepackage{wrapfig}
\usepackage[top=6mm,bottom=8mm,left=4mm,right=4mm]{geometry}
\usepackage[framemethod=tikz]{mdframed}
\usepackage{microtype}
\usepackage{pdfpages}
\usepackage{amsthm,amsmath,amscd}   % Математические дополнения от AMS
\usepackage{amsfonts,amssymb}       % Математические дополнения от AMS
\usepackage{mathtools}              % Добавляет окружение multlined
\usepackage{xfrac}                  % Красивые дроби
\usepackage{physics}

\usepackage{fancyhdr} % колонтитулы

%некоторые математические команды
\newcommand{\Div}{\operatorname{div}}
\newcommand{\Grad}{\operatorname{grad}}

\let\bar\overline

\definecolor{myblue}{cmyk}{1,.72,0,.38}

\def\firstcircle{(0,0) circle (1.5cm)}
\def\secondcircle{(0:2cm) circle (1.5cm)}

\colorlet{circle edge}{myblue}
\colorlet{circle area}{myblue!5}

\tikzset{filled/.style={fill=circle area, draw=circle edge, thick},
	outline/.style={draw=circle edge, thick}}

\pgfdeclarelayer{background}
\pgfsetlayers{background,main}

%\everymath\expandafter{\the\everymath \color{myblue}}
\everydisplay\expandafter{\the\everydisplay \color{myblue}}

\renewcommand{\baselinestretch}{.8}
\pagestyle{empty}

\global\mdfdefinestyle{header}{%
	linecolor=gray,linewidth=1pt,%
	leftmargin=0mm,rightmargin=0mm,skipbelow=0mm,skipabove=0mm,
}

\makeatletter % Author: ttps://tex.stackexchange.com/questions/218587/how-to-set-one-header-for-each-page-using-multicols
\renewcommand{\section}{\@startsection{section}{1}{0mm}%
	{.2ex}%
	{.2ex}%x
	{\color{myblue}\sffamily\small\bfseries}}
\renewcommand{\subsection}{\@startsection{subsection}{1}{0mm}%
	{.2ex}%
	{.2ex}%x
	{\sffamily\bfseries}}

\makeatother
\setlength{\parindent}{0pt}

%колонтитулы
\pagestyle{fancy}
\fancyhf{}
\setlength{\headheight}{40pt}
\setlength{\headsep}{4pt}
\renewcommand{\headrulewidth}{1pt}
\fancyhead[L]{\textcopyright~\colontitulAutors}
\fancyhead[C]{Программа минимум по курсу <<\colontitulEducationalSubject>> \colontitulYear г}
\fancyhead[R]{Преподаватель:~\colontitulTeacher}

\begin{document}
	\small
	\begin{multicols*}{2}
		\section{Определение полупроводника}
		Полупроводникам дается определение как твердым телам, электропроводность которых меняется от $10^{-10}$ до $10^4$ $\text{Ом}^{-1} \text{см}^{-1}$ и которые, таким образом, оказываются классом твердых тел, расположенных между металлами и диэлектриками. Полупроводник имеет кристаллическую
		структуру, т.е. состоит из атомов, упорядоченно расположенных в пространстве.\\
		Полупроводники – материалы, которые с одной стороны пропускают электрический ток, а с другой, в них может существовать электрическое поле, которое может управлять движением электронов.

		\section{Дефекты кристаллической структуры твёрдых тел}
		$\divideontimes$ Вакансия - недостаток атома в кристаллической решетке\\
		$\divideontimes$ Межузельный атом: химических связей у этого атома нет, потому что все валентные электроны он держит при себе. Если межузельный атом встроится в узел вместо какого-либо атома решётки, то либо решётка восстановится (для атомов того же сорта, появление химических связей), либо станет атомом примеси (для атомов другого сорта)\\
		$\divideontimes$ Атом примеси: если количество валентных электронов больше/меньше, чем у остальных атомов решётки, тогда такой атом даёт либо электрон/дырку, которые организуют электрический ток.\\
		$\divideontimes$ Поверхность: так как периодичность решётки на поверхности нарушается то это дефект (протяжённый дефект) и он рассеивает электроны. Поверхностные атомы могут замыкаться друг на друга - будет нарушение периодичности (даже в случае чистой поверхности).\\
		$\divideontimes$ Гетерограница - граница раздела двух кристаллов с различным химическим составом\\
		$\divideontimes$ Дислокации и границы зёрен. Дефекты могут быть стабилизированы отжигом или другим видом диффузии.

		\section{Дырка}
		Незанятые электронами (вакантные) состояния называют дырочными состояниями или просто дырками. Для одной и той же зоны нельзя пользоваться сразу двумя способами описания: если считать, что ток переносят электроны, то незаполненные уровни не дают в него никакого вклада; если же считать, что ток переносят дырки, то отсутствует вклад от электронов.

		\section{Собственный и примесный полупроводник}
		Собственный полупроводник - для которого влияние примесных атомов не существенно. Свободные носители заряда в этом случае возникают только за счет разрыва валентных связей, поэтому число дырок равно числу электронов: $n = p = n_i$. Уровень Ферми $W_f$ собственного полупроводника при абсолютном нуле
		температуры лежит в центре запрещенной зоны и, вообще говоря, смещается при возрастании температуры.\\ 
		Если в полупроводник введены атомы примеси, то он называется примесным. Для определенности будем считать, что полупроводник легирован донорной примесью. Тогда для сохранения электронейтральности отрицательный заряд электронов должен быть равен полному заряду дырок и ионизованных доноров: $n = N_d + p$. Введение атомов донорной примеси приводит к появлению в запрещенной зоне дополнительных локальных энергетических уровней, которые при абсолютном нуле заполнены электронами. Эти уровни соответствуют электронам, которые не создали валентных связей, из-за того, что атом примеси (донор) имеет на внешней электронной оболочке на один или несколько электронов больше, чем у атомов полупроводника.\\
		Для более высоких температур, когда вся примесь ионизована, но вероятность перехода электронов из валентной зоны мала, концентрация носителей
		заряда просто равна концентрации примеси $n=N_d$. Для широко используемых в микроэлектронике полупроводников типа Si, Ge, GaAs, InP, GaAlAs и т.п. соотношение $n=N_d$ реализуется при комнатной температуре. Это позволяет управлять концентрацией электронов в рабочих областях полупроводниковых приборов за счет изменения концентрации атомов легирующей примеси при изготовлении полупроводниковых структур.\\
		Собственная концентрация $n_i$ носителей заряда в полупроводнике называется концентрация, которая определена тепловой генерацией зона-зона в идеальной кристаллической решеткой.

		\section{Уровень Ферми}
		Вероятность того, что в тепловом равновесии при температуре $Т$ состояние с энергией $W$ занято электроном, описывается функцией Ферми–Дирака: \\
		$f(W) = \dfrac{1}{exp(\dfrac{W-W_f}{k_BT})}+1$, где $k_B$ – постоянная Больцмана, $Т$ – абсолютная температура, $W_f$ – энергия (уровень) Ферми. Энергия Ферми – это максимальная энергия электронов при абсолютном нуле температуры в идеальном электронном газе. Зависимость уровня Ферми от температуры в примесных полупроводниках значительно сложнее, чем в собственных, так как определяется не только плотностями состояний в зонах, но и концентрацией примесей.\\
		С точки зрения распределения электронного газа в вакууме уровень Ферми - постоянная величина, не зависящая от температуры. В полупроводнике же уровень Ферми от температуры зависит

		\section{Ширина запрещенной зоны}
		Ширина запрещённой зоны $w_g$ – это энергия, которая требуется, чтобы разорвать валент-ную связь, т.е. электрон сможет перейти из валентной зоны (ВЗ) в зону проводимости (ЗП). Такой электрон перестаёт быть связанным с конкретным атомом и становится свободным в пределах кристаллической решётки, в отсутствие поля он хаотически двигается как молекула идеального газа, если же поле есть – получим ток.\\
		Место отрыва электрона называется дыркой и обладает положительным зарядом, ядро атома, по-лучившееся в результате такого отрыва называется ионом. Благодаря обмену валентными электронами дырка может путешествовать по кристаллу, в этом случае электрический ток может протекать и через валентную зону. Дырка не есть ион в прямом понимании, т.к. ядро атома не двигается при перемещении, а дырка, даже несмотря на то, что это тоже положительный заряд, перемещается благодаря обмену валентными электронами

		\section{Примеси}
		Вырожденные п/п – п/п с высокой концентрацией электронов и их особым распределением по энергиям\\
		\textbf{Примесные атомы} – атомы, отличающиеся от атомов полупроводника своей валентностью и/или атомным номером.\\
		\textbf{Акцептор} – атом с меньшей валентностью, чем атом полупроводника.\\
		\textbf{Донор} – атом с большей валентностью, чем атом полупроводника. Они могут поставлять электроны в зону проводимости или дырки в валентную зону.\\
		Из-за того, что примесей много – дно ЗП смещается к потолку ВЗ. Барьер между атомами примеси уменьшается, так что атомы примеси могут обмениваться электронами – возникает примесная зона. Ширина примесной зоны показывает, на сколько расщепились уровни при объединении ям (как только ямы объединяются, то уровни расщепляются).\\
		В общем случае условие электронейтральности примесного п/п записывается как $n + N_A = p + N_d$, где $N_A$ – концентрация ионов акцепторов, $N_d$ – концентрация ионов доноров.

		\section{Время максвелловской релаксации}
		Это время, за которое распадается случайная флуктуация электронов.

		\section{Длина Дебая}
		Длина Дебая – это характерная длина, на которой электронный газ может экранировать электрическое поле встроенных в кристаллическую решетку зарядов, например, ионов легирующей примеси. Иными словами, при своем движении электроны взаимодействуют только с зарядами, находящимися на расстоянии не больше, чем длина Дебая. Для используемых в полупроводниковых приборах концентраций электронов длина Дебая составляет 10 … 1000 нм

		\section{Пьезоэффект}
		Пьезоэффект – возникновение сжатия или растяжения кристалла в электрическом поле, это эффект обратимый. При $E\neq 0$ сильнее сжать кристалл нельзя – атомы отталкиваются друг от друга. Если при $E = 0$ механически сжать полупроводник и получить такую же картинку, что при $E\neq 0$, то на гранях кристалла появится положительный/отрицательный заряд, а внутри кристалла будет электрическое поле. Эффект используется во встречно-штыревом преобразователе. Эквивалентная схема ВШП - колебательный контур с потерями

		\section{Зоны на диаграммах}
		Самая верхняя полностью заполненная электронами зона называется валентной зоной. Ближайшая к ней незаполненная или частично заполненная зона называется зоной проводимости. Электрические и оптические свойства полупроводников обусловлены переходами электронов между этими двумя зонами, так как все зоны, расположенные ниже, обычно заполнены электронами, а выше – пусты. Электроны заполненной валентной зоны не могут участвовать в переносе электрического тока. С точки зрения зонной теории разница между полупроводниками и диэлектриками заключается только в ширине запрещенной зоны ($W_g$). Обычно к полупроводникам относят материал с шириной запрещенной зоны меньше 2 эВ.

		\section{Эффективная масса}
		Эффективная масса - мера инертности электронов и дырок при их разгоне в электрическом поле, которая определяется из дисперсионной характеристики по следующей формуле: $\dfrac{1}{m^*_{ij}} = \dfrac{1}{\hbar} \dfrac{\partial^2 W(k)}{\partial k_i \partial k_j}$. Электрон как пакет электромагнитных волн при движении пропускает сквозь себя атомы, и мы это описываем с помощью эффективной массы. В зависимости от направления тока в полупроводниковых приборах эффективная масса будет разной и ток тоже – имеет место анизотропия.

		\section{Дрейфовый и диффузионный ток}
		Движение носителей заряда в образце полупроводника может возникать, прежде всего, под действием электрического поля. Образующийся электрический ток принято называть дрейфовым. Кроме того, движение носителей может обусловливаться пространственной неоднородностью их концентрации (градиентом концентрации носителей). При этом возникает так называемый диффузионный ток.

		\section{Обратная решетка}
		Пространство волновых векторов k с размерностью $\text{см}^{-1}$, называется пространством обратной решетки. Элементарная ячейка обратной решетки называется ячейкой Вигнера-Зейтца.\\
		Правило построения ячейки Вигнера-Зейтца:\\
		$\divideontimes$ провести диагонали в ячейке прямой решетки, точку пересечения диагоналей выбрать началом координат;\\
		$\divideontimes$ поделить отрезки от центра к вершинам пополам;\\
		$\divideontimes$ через их середины провести перпендикулярные плоскости.\\
		Прямоугольные грани – x грани (6 штук), совпадают с гранями куба; шестиугольные грани – L грани (8 штук) и точка Г (центр координат, он же центр куба). В зависимости от направления осей (ГX или ГL), решение уравнения Шредингера будет различным, так как расстояние до соседних атомов разные в различных направлениях. Применительно к энергетическим спектрам электронов, ячейка Вигнера-Зейтца называется первой зоной Бриллюэна

		\section{Подвижность и фононы}
		Коэффициент пропорциональности $\mu_n$ между скоростью и полем называется подвижностью электронов и определяется как средняя дрейфовая скорость электронов в единичном электрическом поле: $v_\text{дрейф.} = \mu_n E = \dfrac{e \tau ?????????????}{m^*}$. Следует заметить, что эта формула справедлива только для напряженностей электрического поля менее $10^3$ В/см, а для более сильных полей наблюдается насыщение средней скорости электронов и она выходит на постоянное, не зависящее от электрического поля значение $10^7$ В/см. Это объясняется тем, что при таких больших полях электроны обладают достаточной энергией, чтобы эффективно передавать свою энергию оптическим фононам – колебаниям кристаллической решетки, с характерными энергиями около $0.02–0.06$ эВ при комнатной температуре. При меньших полях электрон может возбудить только акустический фонон, который имеет на порядок меньшую энергию.\\
		$\mu = \dfrac{e \tau_\text{максв.релакс.}}{m^*}$

		\section{Соотношение Эйнштейна}
		$D_n = \dfrac{kT\mu_n}{e}$ - значение коэффициента диффузии. \\
		Соотношение выведено для состояния равновесия, оно плохо работает в полях более 1 кВ/см (условие применимости соотношения - $\Delta n L_\text{своб. проб.}\ll n$). Физический смысл соотношения связан с тем, что тепловая компонента скорости у электронов гораздо больше дрейфовой, поэтому характер движения электрона как при диффузии, так и при дрейфе близок к броуновскому со слабым смещением в сторону тока. Если рассматривать равновесие, компоненты направленного движения (диффузионного и дрейфового) уравновешиваются, а броуновское остается

		\section{Проводимость и ток в полупроводнике}
		$\sigma = en\mu = en\mu_n + ep\mu_p$, \quad $\vec{j_n} = en\mu_n\vec{E} + cD_n\Delta n$, \quad $\vec{j_n} = ep\mu_p\vec{E} - cD_p\Delta p$
		
		\section{Локальная и нелокальная связь поля и дрейфовой скорости}
		$\Delta V = 4 \pi e (N_d + p - N_p - n)$ - уравнение Пуассона\\
		$\dfrac{\partial n}{\partial t} = - \dfrac{1}{e} div \mu + G - R$ - уравнение непрерывности (закон сохранения заряда - уравнение Эйнштейна). Изменение концентрации носителей заряда можно выразить через изменение потока, коэффициенты генерации и рекомбинации.\\
		$j_n = en \mu(E) E + e D_n(E) \Delta n$ - от поля зависит коэффициент диффузии, но не сама диффузия. Первое слагаемое - дрейф в $\vec{E}$, второе слагаемое - диффузия

		\section{Уравнения баланса (квазигидродинамическое приближение)}
		$\dfrac{\partial W_e n}{\partial t} = \dfrac{1}{e} div j_W + \vec{j}\vec{E} - \dfrac{n(W_e - W_0)}{\tau_W(W_e)}$ - уравнение баланса энергии. Причины изменения $W_e$:\\
		$\divideontimes$ первое слагаемое – перетекание электронов (втекает больше, чем вытекает)\\
		$\divideontimes$ второе слагаемое – за счет разогрева в электрическом поле\\
		$\divideontimes$ третье слагаемое – потери за счет столкновений с кристаллической решеткой (фононы)\\
		$j_W = en \mu E W_e + e \Delta(D_n n W_e)$ - плотность потока энергии\\
		$\dfrac{\partial m (W_e) \vec{V}}{\partial t} = -e\vec{E} - \dfrac{m\vec{V}}{\tau_p(W_e)}$ - уравнение баланса импульса (сила Кулона и сила со стороны кристаллической решетки) - по 2 закону Ньютона \\
		Если поля меняются плавно по сравнению с $L_w$ и $L_p$, и временами $\tau_w$ и $\tau_p$, то этими уравнениями можно пренебречь (не работает в любых современных п/п приборах)

		\section{Пробой}
		Коэффициент ударной ионизации $\alpha$ – количество электронно-дырочных пар, которое генерирует один электрон на 1 сантиметре пути.\\
		Квантовый выход ($\theta$) – это количество электронно-дырочных пар, образованных одним фотоном\\
		\textbf{Пробой} – резкое увеличение числа электронов и дырок из-за разрыва валентных связей. Причины: импульсные электрические поля, разрыв валентных связей непосредственно электрическим полем, за счет импульса света – оптический пробой, за счет импульса температур (необратимый пробой)\\
		\textbf{Лавинный пробой:} максимальная скорость электрона соответствует энергии больше ширины запрещенной зоны,	тогда при ударе о кристаллическую решетку генерируется не только оптический фонон, но и	разрывается валентная связь.\\
		\textbf{Тепловой пробой:} пусть из-за большого поля возникает лавинный пробой, ток растет, следовательно, растет температура (и газа и решётки). Это приводит к уменьшению ширины ЗЗ (кристаллическую решетку ионизировать легче), следовательно, коэффициент ударной ионизации растет, а значит растёт и ток – итерация повторяется снова и т.д.\\
		Поскольку тепловой пробой расплавляет кристаллическую решетку, он необратим. В области дефектов ионизировать материал легче, поэтому явление пробоя часто возникает вдоль поверхностей, скопления примесей и т.д.

		\section{Эффект всплеска}
		Эффектом всплеска скорости называется резкое увеличение дрейфовой скорости заряда, который наблюдается в резко неоднородных полях с характерным масштабом порядка длин релаксации импульса и энергии

		\section{Механизм рекомбинации и генерации за счет взаимодействия с кристаллической решеткой}
		Кроме взаимодействия с кристаллической решеткой в непрямозонных полупроводниках, когда при переходе зона-зона генерируется фонон, может быть еще рекомбинация через уровни в запрещенной зоне. Если в запрещенной зоне существует энергетический уровень, тогда имеют смысл 4 процесса: эмиссия электрона, захват электрона, эмиссия дырки, захват дырки

	\end{multicols*}
\end{document}

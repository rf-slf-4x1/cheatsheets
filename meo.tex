\documentclass[10pt,landscape,a4paper]{article}
\usepackage[utf8]{inputenc}
\usepackage[english, russian]{babel}
\usepackage[T1,T2A]{fontenc}  
\usepackage{upgreek} % прямые греческие ради русской традиции
\usepackage{tikz}
\usetikzlibrary{shapes,positioning,arrows,fit,calc,graphs,graphs.standard}
%\usepackage[nosf]{kpfonts}
%\usepackage[t1]{sourcesanspro}
\usepackage{multicol}
\usepackage{wrapfig}
\usepackage[top=6mm,bottom=8mm,left=4mm,right=4mm]{geometry}
\usepackage[framemethod=tikz]{mdframed}
\usepackage{microtype}
\usepackage{pdfpages}
\usepackage{amsthm,amsmath,amscd}   % Математические дополнения от AMS
\usepackage{amsfonts,amssymb}       % Математические дополнения от AMS
\usepackage{mathtools}              % Добавляет окружение multlined
\usepackage{xfrac}                  % Красивые дроби
\usepackage{physics}

\usepackage{fancyhdr} % колонтитулы

%некоторые математические команды
\newcommand{\Div}{\operatorname{div}}
\newcommand{\Grad}{\operatorname{grad}}

\let\bar\overline

\definecolor{myblue}{cmyk}{1,.72,0,.38}

\def\firstcircle{(0,0) circle (1.5cm)}
\def\secondcircle{(0:2cm) circle (1.5cm)}

\colorlet{circle edge}{myblue}
\colorlet{circle area}{myblue!5}

\tikzset{filled/.style={fill=circle area, draw=circle edge, thick},
	outline/.style={draw=circle edge, thick}}

\pgfdeclarelayer{background}
\pgfsetlayers{background,main}

%\everymath\expandafter{\the\everymath \color{myblue}}
\everydisplay\expandafter{\the\everydisplay \color{myblue}}

\renewcommand{\baselinestretch}{.8}
\pagestyle{empty}

\global\mdfdefinestyle{header}{%
	linecolor=gray,linewidth=1pt,%
	leftmargin=0mm,rightmargin=0mm,skipbelow=0mm,skipabove=0mm,
}

\makeatletter % Author: ttps://tex.stackexchange.com/questions/218587/how-to-set-one-header-for-each-page-using-multicols
\renewcommand{\section}{\@startsection{section}{1}{0mm}%
	{.2ex}%
	{.2ex}%x
	{\color{myblue}\sffamily\small\bfseries}}
\renewcommand{\subsection}{\@startsection{subsection}{1}{0mm}%
	{.2ex}%
	{.2ex}%x
	{\sffamily\bfseries}}

\makeatother
\setlength{\parindent}{0pt}

%колонтитулы
\pagestyle{fancy}
\fancyhf{}
\setlength{\headheight}{40pt}
\setlength{\headsep}{4pt}
\renewcommand{\headrulewidth}{1pt}
\fancyhead[L]{\textcopyright~\colontitulAutors}
\fancyhead[C]{Программа минимум по курсу <<\colontitulEducationalSubject>> \colontitulYear г}
\fancyhead[R]{Преподаватель:~\colontitulTeacher}

\newcommand{\sumk}{\sum\limits_{k=1}^3v_k}

\begin{document}
	\small
	\begin{multicols*}{4}
		\section{Понятие субстанциальной и локальной производных.}
		$\dfrac{d}{dt}=\dfrac{\partial}{\partial t}+(\vec{v}\nabla)$ - субстанциальная \\
		$\dfrac{\partial}{\partial t}$~-~локальная
		
		\section{Уравнение неразрывности для сжимаемой и несжимаемой жидкости.}
		$\dfrac{d\rho}{dt}+\rho \div(\vec{v})=0$, \\
		$\dfrac{d\rho}{dt}=0$ для несжимаемой 
		
		\section{Уравнение Эйлера в векторной форме и в проекциях на оси в декартовой системе координат.}
		$\frac{d\vec{v}}{dt}=-\frac{\nabla p}{\rho}+\vec{f}$ \\
		$\dfrac{\partial v_i}{\partial t}+\sumk\dfrac{\partial v_i}{\partial x_k}=-\dfrac{1}{\rho}\dfrac{\partial p}{\partial t}+f_i$
		
		\section{Закон сохранения энергии идеальной жидкости. Поток энергии.}
		$\int\limits_V=\left[\dfrac{\partial}{\partial t}(\dfrac{\rho v^2}{2}+\rho\varepsilon)+\div(\dfrac{\rho v^2}{2}+W)\vec(v)\right]dV=0$, где\\
		$W=\rho\varepsilon+p$ - энтальпия \\
		или в дифференциальной форме \\
		$\dfrac{\partial E}{\partial t}+div\vec{E}=0$, где \\
		$E=\dfrac{\rho v^2}{2}+\rho\varepsilon$ - плотность энергии \\
		$\vec{N}=\left[\dfrac{\rho v^2}{2}+\rho\varepsilon+p\right]\vec{v}$ - вектор плотностьи потока энергии
		
		\section{Закон сохранения импульса идеальной жидкости.Тензор плотности потока импульса и его представление в декартовой системе координат.}
		$ \dfrac{\partial}{\partial t}\int\limits_V p\vec(v)dV=-\oint\limits_S \left[p\vec{n}+\rho\vec{v}(\vec{v}\vec{n})\right]d\sigma$ \\
		$ \dfrac{\partial}{\partial t}(\rho v_i)=-\sumk\dfrac{\partial \Pi_{ik}}{\partial x_k}+\rho f_i $ \\
		$ \Pi_{ik} = p\delta_{ik}+\rho v_iv_k$ - тензор ППИ
		
		\section{Уравнение гидростатики.}
		$\frac{d\vec{v}}{dt}=-\frac{\nabla p}{\rho}+\vec{f}$ - Эйлера\\
		$\dfrac{\partial\rho}{\partial t}+\div(\rho\vec{v})=0$ \\
		$p=p(\rho)$
		
		\section{Частота Брента-Вяйсяля.}
		$N=\sqrt{\dfrac{g}{\rho}\dfrac{d\rho}{dz}}$
		
		\section{Теорема Бернулли для потенциальных и непотенциальных, стационарных и нестационарных течений.}
		$\dfrac{v^2}{2}+\dfrac{p}{\rho}-gz=const$ - потенциальное \\
		$\dfrac{v^2}{2}+W-gz=const$ - стационарное \\
		$\dfrac{\partial \varphi}{\partial t}+\dfrac{v^2}{2}+\dfrac{p}{\rho}-gz=const$ - нестационарное не вихревое \\
		\textbf{не полностью, проверить!!!}
		
		\section{Теорема Томсона.}
		Циркуляция скорости вдоль замкнутого контура, перемещающегося в идеальной жидкости, остается постоянной. \\
		$\dfrac{d\Gamma}{dt}=\oint\limits_Ld\left(\dfrac{v^2}{2}-W-u\right)=0$ \\
		$\Gamma = \oint\limits_L\vec{v}d\vec{r}$ - циркуляция
		
		\section{Потенциальные течения идеальной несжимаемой жидкости. Основные уравнения, граничные условия.}
		$\Delta \varphi=0,~\vec{v}=\grad(\phi)$ \\
		граничное словие непроникания: \\
		$\vec{v}\vec{n}|_s=\dfrac{\partial\varphi}{\partial n}=\vec{v_0}\vec{n}$ \\
		\textbf{Граничное условие на бесконечности?}
		
		\section{Парадокс Д’Аламбера-Эйлера.}
		Формулировка 1. При обтекании тела с гладкой поверхностью идеальной
		несжимаемой жидкостью сила лобового сопротивления, действующая на тело
		со стороны потока, равна нулю. \\
		Формулировка 2. Для тела, движущегося равномерно в идеальной несжимаемой жидкости постоянной плотности без границ, сила сопротивления равна
		нулю.\\
		$\vec{F}=-\oint\limits_sp_s\vec{n}dS=0$
		
		\section{Понятие присоединенной массы.Присоединенная масса сферы и единицы длины бесконечного кругового цилиндра.}
		$F-F_\text{сопр}=ma$ \\
		$M=F_\text{сопр}/a$ \\
		$F=(M+m)a$ \\ 
		\textbf{тут могли быть присоеденненые масы сферы и цилиндра, но нету)}
		
		\section{Функция тока и ее свойства.}
		\textbf{хз что тут хотят}
		
		\section{Комплексный потенциал.}
		$F(z)=\phi+i\Psi$ (действительная часть
		потенциал, мнимая – функция тока)
		
		\section{Линии тока и эквипотенциальные линии.}
		$\Psi = const$ - линии тока (постоянная функция тока ) \\
		$\varphi = const$ - эквипотенциальные линии (постоянный потенциал)
		
		\section{Формула Жуковского.}
		$F_y=\int p_nydl=\rho\Gamma v_0$
		
		\section{Точечные вихри и их взаимодействия.}
		\textbf{хз что тут хотят}
		
		\section{Поверхностные гравитационные волны (длинные, короткие, гравитационно-капиллярные) и их основные свойства (траектории движения частиц, дисперсионные уравнения, фазовые и групповые скорости).}
		\section{Уравнение Навье-Стокса для несжимаемой вязкой жидкости в векторной форме и в проекциях на оси в декартовой системе координат.}
		\section{Тензор вязких напряжений, физический смысл, представление в декартовой системе координат.}
		\section{Граничные условия для несжимаемой вязкой жидкости на поверхности твердого тела и свободной поверхности.}
		\section{Формула Пуазейля для расхода жидкости.}
		\section{Скин-слой.}
		\section{Числа Рейнольдса, Фруда, Струхаля и их физический смысл.}
		\section{Формула Стокса.}
		\section{Зависимость ширины пограничного слоя от параметров.}
		\section{Уравнения линейной акустики.Волновое уравнение.}
		\section{Монохроматические волны, уравнение Гельмгольца}
		\section{Закон сохранения энергии (звуковой волны)}
	\end{multicols*}
\end{document}
\documentclass[10pt,landscape,a4paper]{article}
\usepackage[utf8]{inputenc}
\usepackage[russian]{babel}
\usepackage[T1]{fontenc}
\usepackage{tikz}
\usetikzlibrary{shapes,positioning,arrows,fit,calc,graphs,graphs.standard}
\usepackage[nosf]{kpfonts}
\usepackage[t1]{sourcesanspro}
\usepackage{multicol}
\usepackage{wrapfig}
\usepackage[top=0mm,bottom=1mm,left=0mm,right=1mm]{geometry}
\usepackage[framemethod=tikz]{mdframed}
\usepackage{microtype}
\usepackage{pdfpages}

\let\bar\overline

\documentclass[10pt,landscape,a4paper]{article}
\usepackage[utf8]{inputenc}
\usepackage[english, russian]{babel}
\usepackage[T1,T2A]{fontenc}  
\usepackage{upgreek} % прямые греческие ради русской традиции
\usepackage{tikz}
\usetikzlibrary{shapes,positioning,arrows,fit,calc,graphs,graphs.standard}
%\usepackage[nosf]{kpfonts}
%\usepackage[t1]{sourcesanspro}
\usepackage{multicol}
\usepackage{wrapfig}
\usepackage[top=6mm,bottom=8mm,left=4mm,right=4mm]{geometry}
\usepackage[framemethod=tikz]{mdframed}
\usepackage{microtype}
\usepackage{pdfpages}
\usepackage{amsthm,amsmath,amscd}   % Математические дополнения от AMS
\usepackage{amsfonts,amssymb}       % Математические дополнения от AMS
\usepackage{mathtools}              % Добавляет окружение multlined
\usepackage{xfrac}                  % Красивые дроби
\usepackage{physics}

\usepackage{fancyhdr} % колонтитулы

%некоторые математические команды
\newcommand{\Div}{\operatorname{div}}
\newcommand{\Grad}{\operatorname{grad}}

\let\bar\overline

\definecolor{myblue}{cmyk}{1,.72,0,.38}

\def\firstcircle{(0,0) circle (1.5cm)}
\def\secondcircle{(0:2cm) circle (1.5cm)}

\colorlet{circle edge}{myblue}
\colorlet{circle area}{myblue!5}

\tikzset{filled/.style={fill=circle area, draw=circle edge, thick},
	outline/.style={draw=circle edge, thick}}

\pgfdeclarelayer{background}
\pgfsetlayers{background,main}

%\everymath\expandafter{\the\everymath \color{myblue}}
\everydisplay\expandafter{\the\everydisplay \color{myblue}}

\renewcommand{\baselinestretch}{.8}
\pagestyle{empty}

\global\mdfdefinestyle{header}{%
	linecolor=gray,linewidth=1pt,%
	leftmargin=0mm,rightmargin=0mm,skipbelow=0mm,skipabove=0mm,
}

\makeatletter % Author: ttps://tex.stackexchange.com/questions/218587/how-to-set-one-header-for-each-page-using-multicols
\renewcommand{\section}{\@startsection{section}{1}{0mm}%
	{.2ex}%
	{.2ex}%x
	{\color{myblue}\sffamily\small\bfseries}}
\renewcommand{\subsection}{\@startsection{subsection}{1}{0mm}%
	{.2ex}%
	{.2ex}%x
	{\sffamily\bfseries}}

\makeatother
\setlength{\parindent}{0pt}

%колонтитулы
\pagestyle{fancy}
\fancyhf{}
\setlength{\headheight}{40pt}
\setlength{\headsep}{4pt}
\renewcommand{\headrulewidth}{1pt}
\fancyhead[L]{\textcopyright~\colontitulAutors}
\fancyhead[C]{Программа минимум по курсу <<\colontitulEducationalSubject>> \colontitulYear г}
\fancyhead[R]{Преподаватель:~\colontitulTeacher}

\begin{document}
	\small
	\begin{multicols*}{5}
		\section{Понятие субстанциальной и локальной производных}
		\section{Уравнение Эйлера в векторной форме и в проекциях на оси в декартовой системе координат}
		\section{Закон сохранения энергии идеальной жидкости}
		\section{Поток энергии}
		\section{Закон сохранения импульса идеальной жидкости}
		\section{Тензор плотности потока импульса и его представление в декартовой системе координат}
		\section{Уравнение гидростатики}
		\section{Частота Брента-Вяйсяля}
		\section{Теорема Бернулли для потенциальных и непотенциальных, стационарных и нестационарных течений}
		\section{Теорема Томсона}
		\section{Потенциальные течения идеальной несжимаемой жидкости. Основные уравнения, граничные условия}
		\section{Парадокс Д’Аламбера-Эйлера}
		\section{Понятие присоединенной массы}
		\section{Присоединенная масса сферы и единицы длины бесконечного кругового цилиндра}
		\section{Функция тока и ее свойства}
		\section{Комплексный потенциал}
		\section{Линии тока и эквипотенциальные линии}
		\section{Формула Жуковского}
		\section{Точечные вихри и их взаимодействия}
		\section{Поверхностные гравитационные волны (длинные, короткие, гравитационно-капиллярные) и их основные свойства (траектории движения частиц, дисперсионные уравнения, фазовые и групповые скорости)}
		\section{Уравнение Навье-Стокса для несжимаемой вязкой жидкости в векторной форме и в проекциях на оси в декартовой системе координат}
		\section{Тензор вязких напряжений, физический смысл, представление в декартовой системе координат}
		\section{Граничные условия для несжимаемой вязкой жидкости на поверхности твердого тела и свободной поверхности}
		\section{Формула Пуазейля для расхода жидкости}
		\section{Скин-слой}
		\section{Числа Рейнольдса, Фруда, Струхаля и их физический смысл}
		\section{Формула Стокса}
		\section{Зависимость ширины пограничного слоя от параметров}
		\section{Уравнения линейной акустики}
		\section{Волновое уравнение}
		\section{Монохроматические волны, уравнение Гельмгольца}
		\section{Закон сохранения энергии (звуковой волны)}
	\end{multicols*}
\end{document}
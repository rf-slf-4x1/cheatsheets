% Подписи колонтитула
\newcommand{\colontitulAutors}{edombek}
\newcommand{\colontitulYear}{2022}
\newcommand{\colontitulEducationalSubject}{Прикладная электродинамика}
\newcommand{\colontitulTeacher}{Гиндельбург~В.~Б.}

%Настройки шаблона
\documentclass[10pt,landscape,a4paper]{article}
\usepackage[utf8]{inputenc}
\usepackage[english, russian]{babel}
\usepackage[T1,T2A]{fontenc}  
\usepackage{upgreek} % прямые греческие ради русской традиции
\usepackage{tikz}
\usetikzlibrary{shapes,positioning,arrows,fit,calc,graphs,graphs.standard}
%\usepackage[nosf]{kpfonts}
%\usepackage[t1]{sourcesanspro}
\usepackage{multicol}
\usepackage{wrapfig}
\usepackage[top=6mm,bottom=8mm,left=4mm,right=4mm]{geometry}
\usepackage[framemethod=tikz]{mdframed}
\usepackage{microtype}
\usepackage{pdfpages}
\usepackage{amsthm,amsmath,amscd}   % Математические дополнения от AMS
\usepackage{amsfonts,amssymb}       % Математические дополнения от AMS
\usepackage{mathtools}              % Добавляет окружение multlined
\usepackage{xfrac}                  % Красивые дроби
\usepackage{physics}

\usepackage{fancyhdr} % колонтитулы

%некоторые математические команды
\newcommand{\Div}{\operatorname{div}}
\newcommand{\Grad}{\operatorname{grad}}

\let\bar\overline

\definecolor{myblue}{cmyk}{1,.72,0,.38}

\def\firstcircle{(0,0) circle (1.5cm)}
\def\secondcircle{(0:2cm) circle (1.5cm)}

\colorlet{circle edge}{myblue}
\colorlet{circle area}{myblue!5}

\tikzset{filled/.style={fill=circle area, draw=circle edge, thick},
	outline/.style={draw=circle edge, thick}}

\pgfdeclarelayer{background}
\pgfsetlayers{background,main}

%\everymath\expandafter{\the\everymath \color{myblue}}
\everydisplay\expandafter{\the\everydisplay \color{myblue}}

\renewcommand{\baselinestretch}{.8}
\pagestyle{empty}

\global\mdfdefinestyle{header}{%
	linecolor=gray,linewidth=1pt,%
	leftmargin=0mm,rightmargin=0mm,skipbelow=0mm,skipabove=0mm,
}

\makeatletter % Author: ttps://tex.stackexchange.com/questions/218587/how-to-set-one-header-for-each-page-using-multicols
\renewcommand{\section}{\@startsection{section}{1}{0mm}%
	{.2ex}%
	{.2ex}%x
	{\color{myblue}\sffamily\small\bfseries}}
\renewcommand{\subsection}{\@startsection{subsection}{1}{0mm}%
	{.2ex}%
	{.2ex}%x
	{\sffamily\bfseries}}

\makeatother
\setlength{\parindent}{0pt}

%колонтитулы
\pagestyle{fancy}
\fancyhf{}
\setlength{\headheight}{40pt}
\setlength{\headsep}{4pt}
\renewcommand{\headrulewidth}{1pt}
\fancyhead[L]{\textcopyright~\colontitulAutors}
\fancyhead[C]{Программа минимум по курсу <<\colontitulEducationalSubject>> \colontitulYear г}
\fancyhead[R]{Преподаватель:~\colontitulTeacher}

\begin{document}
	\small
	\begin{multicols*}{2}
		\section{Запись функции, определяющей зависимость полей и векторных потенциалов гармонической плоской волны в линии передачи от времени $t$ и продольной координаты $z$. Понятия частоты, временного периода, продольного волнового числа, длины волны, фазовой и групповой скорости.}
		
		$\{\vec{E},\vec{H}\}=\{\vec{E_0},\vec{H_0}\}e^{i(wt-hz)}, ~\vec{A}^{e,m}=\vec{z_0}\psi^{e,m}(r_{\perp})e^{-ihz}$ \\
		$\psi$~-~произвольная скалярная функция(амплитуда векторного потенциала), $(wt-hz)$~-~фаза \\
		$\varkappa^2=k^2-h^2$~-~поперечное волновое число, $k=\frac wc\sqrt{\mu\varepsilon}$~-~волноевое число в среде, $ h $~-~ продольное волновое число. \\
		$T=\frac {2\pi}w, \lambda_\text{в}=\frac {2\pi}h, V_\text{ф}=\frac{w}{h}, V_\text{гр}=\dv{w}{h}$. Для волновода без заполнения $V_\text{ф}V_\text{гр}=c^2$. $V_\text{гр}\le c$.
		
		\section{Волновое уравнение для векторного потенциала в отсутствие источников при произвольной и гармонической зависимости от времени. Дифференциальное уравнение для скалярных поперечных волновых функций $\Psi^{(e),(m)}(r_\perp)$, определяющих зависимость полей в линии передачи от поперечных координат. Понятие поперечного волнового числа. }
		
		\section{Понятие о ТЕ, ТМ и ТЕМ волнах. Импедансная связь поперечных компонент полей. Определение поперечного волнового импеданса. }
		
		\section{Граничные условия для полей и поперечных волновых функций $\Psi^{(e)}$ и $\Psi^{(m)}$ в линиях передачи с идеально проводящими границами. Математическая формулировка задачи отыскания собственных волн различных типов в идеальной линии.}
		
		\section{Дисперсионное уравнение для волн в идеальных линиях. Понятие критической частоты и критической длины волны. Графики зависимости полей от продольной координаты в различные моменты времени при частотах, больших или меньших критической. Зависимости длины волны, фазовой и групповой скорости в линии передачи от частоты.}
		
		\section{В каких линиях могут существовать главные (ТЕМ) волны? Поля ТЕМ волны в коаксиальной линии (форма силовых линий и зависимость от координат).}
		
		\section{Спектр поперечных волновых чисел прямоугольного волновода. Низшая мода (поперечное волновое число, графики поля, картина силовых линий). Низшая мода круглого волновода (поперечное волновое число, картина силовых линий).}
		
		\section{Причины затухания волн в линиях передачи. Описание затухания, обусловленного потерями энергии в заполняющей среде. Графики зависимости поля в линии передачи с потерями от продольной координаты в различные моменты времени.}
		
		\section{Описание главных волн в линиях передачи в терминах тока и напряжения: определения величин тока и напряжения, погонной емкости и индуктивности, определения волнового сопротивления, импеданса нагрузки, импеданса в любом сечении линии с произвольной нагрузкой на конце.}
		
		\section{Коэффициент отражения волны от нагрузки на конце линии. Понятие согласования линии с нагрузкой.}
		
		\section{Спектр собственных частот идеального прямоугольного резонатора. Низшая мода прямоугольного резонатора (собственная частота, структура поля).}
		
		\section{Причины затухания колебаний в реальных резонаторах. Описание затухания, обусловленного потерями энергии в заполняющей среде. График зависимости поля собственного колебания в реальном резонаторе от времени.}
		
		\section{Представление полей, создаваемых в волноводе заданными сторонними токами, в виде суперпозиции полей собственных мод (общий вид формул возбуждения волноводов).}
		
		\section{Представление полей, создаваемых в резонаторе заданными сторонними токами, в виде суперпозиции полей собственных колебаний (общий вид формул возбуждения резонатора). Резонансные свойства полей. }
		
		\section{Способы возбуждения волноводов и резонаторов при помощи штыря и петли.}
		
		\section{Определения дифференциального и полного сечений рассеяния тела. Выражение для амплитуды поля и плотности потока энергии рассеянной волны в дальней зоне через дифференциальное сечение рассеяния. }
		
		\section{Приближение геометрической оптики и условия его применимости в задачах дифракции плоской волны на теле. Понятие луча и лучевой трубки. }
		
	\end{multicols*}
\end{document}
% Подписи колонтитула
\newcommand{\colontitulAutors}{astronom\_v\_cube}
\newcommand{\colontitulYear}{2023~}
\newcommand{\colontitulEducationalSubject}{Физика волновых процессов}
\newcommand{\colontitulTeacher}{Петров Е. Ю.}

%Настройки шаблона
\documentclass[10pt,landscape,a4paper]{article}
\usepackage[utf8]{inputenc}
\usepackage[english, russian]{babel}
\usepackage[T1,T2A]{fontenc}  
\usepackage{upgreek} % прямые греческие ради русской традиции
\usepackage{tikz}
\usetikzlibrary{shapes,positioning,arrows,fit,calc,graphs,graphs.standard}
%\usepackage[nosf]{kpfonts}
%\usepackage[t1]{sourcesanspro}
\usepackage{multicol}
\usepackage{wrapfig}
\usepackage[top=6mm,bottom=8mm,left=4mm,right=4mm]{geometry}
\usepackage[framemethod=tikz]{mdframed}
\usepackage{microtype}
\usepackage{pdfpages}
\usepackage{amsthm,amsmath,amscd}   % Математические дополнения от AMS
\usepackage{amsfonts,amssymb}       % Математические дополнения от AMS
\usepackage{mathtools}              % Добавляет окружение multlined
\usepackage{xfrac}                  % Красивые дроби
\usepackage{physics}

\usepackage{fancyhdr} % колонтитулы

%некоторые математические команды
\newcommand{\Div}{\operatorname{div}}
\newcommand{\Grad}{\operatorname{grad}}

\let\bar\overline

\definecolor{myblue}{cmyk}{1,.72,0,.38}

\def\firstcircle{(0,0) circle (1.5cm)}
\def\secondcircle{(0:2cm) circle (1.5cm)}

\colorlet{circle edge}{myblue}
\colorlet{circle area}{myblue!5}

\tikzset{filled/.style={fill=circle area, draw=circle edge, thick},
	outline/.style={draw=circle edge, thick}}

\pgfdeclarelayer{background}
\pgfsetlayers{background,main}

%\everymath\expandafter{\the\everymath \color{myblue}}
\everydisplay\expandafter{\the\everydisplay \color{myblue}}

\renewcommand{\baselinestretch}{.8}
\pagestyle{empty}

\global\mdfdefinestyle{header}{%
	linecolor=gray,linewidth=1pt,%
	leftmargin=0mm,rightmargin=0mm,skipbelow=0mm,skipabove=0mm,
}

\makeatletter % Author: ttps://tex.stackexchange.com/questions/218587/how-to-set-one-header-for-each-page-using-multicols
\renewcommand{\section}{\@startsection{section}{1}{0mm}%
	{.2ex}%
	{.2ex}%x
	{\color{myblue}\sffamily\small\bfseries}}
\renewcommand{\subsection}{\@startsection{subsection}{1}{0mm}%
	{.2ex}%
	{.2ex}%x
	{\sffamily\bfseries}}

\makeatother
\setlength{\parindent}{0pt}

%колонтитулы
\pagestyle{fancy}
\fancyhf{}
\setlength{\headheight}{40pt}
\setlength{\headsep}{4pt}
\renewcommand{\headrulewidth}{1pt}
\fancyhead[L]{\textcopyright~\colontitulAutors}
\fancyhead[C]{Программа минимум по курсу <<\colontitulEducationalSubject>> \colontitulYear г}
\fancyhead[R]{Преподаватель:~\colontitulTeacher}

\begin{document}
	\small
	\begin{multicols*}{2}
		\section{Плоская монохроматическая волна}

		\section{Волновое уравнение}
		$\nabla U - \dfrac{1}{c^2} \dfrac{\partial^2 \vec{U}}{\partial t^2} = 0$ - волновое уравнение без поглощения\\
		$\nabla U - \beta\dfrac{\partial\vec{U}}{\partial t} -\dfrac{1}{c^2} \dfrac{\partial^2 \vec{U}}{\partial t^2} = 0$ - волновое уравнение c поглощением\\
		$U$ - компонента электрического поля / магнитоного поля / скорость / потенциал, $c$ - имеет смысл фазовой скорости\\
		Решение - в виде плоской монохроматической волны $U = U_0 e^{(i\omega t- i k \vec{r})}$, если выполнено $\dfrac{\omega^2}{k^2} = c^2$

		\section{Фазовая и групповая скорости}

		\section{Уравнение непрерывности и уравнение Эйлера}

		\section{Скорость звука. Вектор Умнова. Плотность энергии в звуковой волне}

		\section{Уравнение Ламэ}
		$\rho_0 \dfrac{\partial^2 \vec{U}}{\partial t^2} = (\lambda + \mu) \nabla div \vec{U} + \mu \bigtriangleup  \vec{U}$ - уравнение движения физически бесконечно малого объема изотропного (движение в любых направлениях) упругого тела при малых деформациях\\
		$\rho_0$ - плотность до деформации, $\mu$ - модуль сдвига, $\lambda = K - \frac{2}{3}\mu$ - коэффициент Ламэ, $K$ - модуль всестороннего сжатия, $\vec{U}(\vec{r}, t)$ - вектор смещения элемента сплошной среды при деформации\\
		$\mu$ и $K$ - переобозначения модулей упругости Юнга и Пуассона

		\section{Уравнения Максвелла в дифференциальной и интегральной формах}

		\section{Граничные условия для векторов ЭМ поля}

		\section{Вектор Пойнтинга. Плотность энергии ЭМ поля в вакууме}

		$S = \dfrac{c}{4\pi} \left[\vec{E}\times \vec{H}\right]$ - плотность потока энергии \quad СГС: $\left[\dfrac{\text{эрг}}{\text{c}\cdot \text{см}^2}\right]$ \quad СИ: $\left[\dfrac{\text{Дж}}{\text{c}\cdot \text{м}^2}\right]$\\
		$\left\lvert S\right\rvert$ - энергия, переносимая ЭМ волной через единичную площадку $(\bot S)$ в единицу времени\\
		???

		\section{Основные параметры плазмы (плазменная частота и дебаевский радиус)}
		$r_{De} = \sqrt{\dfrac{k T_e T_i}{4\pi N e^2(T_e + T_i)}} = \sqrt{\dfrac{k T}{4\pi N e^2}}$ - расстояние, за которое волна спадет в $e$ раз при прохождении через плазму / расстояние, которое проходит $\bar{e}$ в плазме за время, порядка $\tau_p = \frac{2\pi}{\omega_p}$\\
		СИ: $\left[{\text{K}}\dot{\text{Дж}}\right]$
		$T_e$ - температура электронов, $T_i$ - температура ионов, $N$, $e$ и $m$ - концетрация электронов а также их заряд и масса, $k = \dfrac{R}{N_a}, N_a = \dfrac{m}{M}$\\
		???\\
		$\omega_p = \dfrac{4\pi e^2 N}{m}$ - плазменная частота, СИ: $\left[\dfrac{\text{рад}}{\text{c}}\right]$ ???\\
		Это частота собственных продольных колебаний пространственного заряда в однородной плазме в отсутствие магнитного поля

		\section{Комплексная диэлектрическая проницаемость холодной изотропной плазмы}
		Диэлектрическая проницаемость показывает, во сколько раз сила взаимодействия двух электрических зарядов в конкретной среде меньше, чем в вакууме, для которого она равна $1$\\
		$\mathcal{E} (\omega) = 1 - \dfrac{\omega_{pe}^2}{\omega(\omega - i\nu_e)} - \chi $, где $\chi = \dfrac{\omega_{pi}^2}{\omega(\omega - i\nu_i)}$ - ионная составляющая, которой можно пренебречь\\
		Вводятся абсолютная $(\mathcal{E}_a)$ и относительная $(\mathcal{E}_r)$ проницаемости. Величина $\mathcal{E} _{r}$ безразмерна, а ${\displaystyle \mathcal{E} _{a}}$ по размерности совпадает с электрической постоянной $\mathcal{E}_{0}$ - СИ: $\left[\dfrac{\text{фарад}}{\text{м}}\right]$


	\end{multicols*}
\end{document}

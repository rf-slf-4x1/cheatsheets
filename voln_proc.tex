% Подписи колонтитула
\newcommand{\colontitulAutors}{astronom\_v\_cube,~edombek}
\newcommand{\colontitulYear}{2023~}
\newcommand{\colontitulEducationalSubject}{Физика волновых процессов}
\newcommand{\colontitulTeacher}{Петров Е. Ю.}

%Настройки шаблона
\documentclass[10pt,landscape,a4paper]{article}
\usepackage[utf8]{inputenc}
\usepackage[english, russian]{babel}
\usepackage[T1,T2A]{fontenc}  
\usepackage{upgreek} % прямые греческие ради русской традиции
\usepackage{tikz}
\usetikzlibrary{shapes,positioning,arrows,fit,calc,graphs,graphs.standard}
%\usepackage[nosf]{kpfonts}
%\usepackage[t1]{sourcesanspro}
\usepackage{multicol}
\usepackage{wrapfig}
\usepackage[top=6mm,bottom=8mm,left=4mm,right=4mm]{geometry}
\usepackage[framemethod=tikz]{mdframed}
\usepackage{microtype}
\usepackage{pdfpages}
\usepackage{amsthm,amsmath,amscd}   % Математические дополнения от AMS
\usepackage{amsfonts,amssymb}       % Математические дополнения от AMS
\usepackage{mathtools}              % Добавляет окружение multlined
\usepackage{xfrac}                  % Красивые дроби
\usepackage{physics}

\usepackage{fancyhdr} % колонтитулы

%некоторые математические команды
\newcommand{\Div}{\operatorname{div}}
\newcommand{\Grad}{\operatorname{grad}}

\let\bar\overline

\definecolor{myblue}{cmyk}{1,.72,0,.38}

\def\firstcircle{(0,0) circle (1.5cm)}
\def\secondcircle{(0:2cm) circle (1.5cm)}

\colorlet{circle edge}{myblue}
\colorlet{circle area}{myblue!5}

\tikzset{filled/.style={fill=circle area, draw=circle edge, thick},
	outline/.style={draw=circle edge, thick}}

\pgfdeclarelayer{background}
\pgfsetlayers{background,main}

%\everymath\expandafter{\the\everymath \color{myblue}}
\everydisplay\expandafter{\the\everydisplay \color{myblue}}

\renewcommand{\baselinestretch}{.8}
\pagestyle{empty}

\global\mdfdefinestyle{header}{%
	linecolor=gray,linewidth=1pt,%
	leftmargin=0mm,rightmargin=0mm,skipbelow=0mm,skipabove=0mm,
}

\makeatletter % Author: ttps://tex.stackexchange.com/questions/218587/how-to-set-one-header-for-each-page-using-multicols
\renewcommand{\section}{\@startsection{section}{1}{0mm}%
	{.2ex}%
	{.2ex}%x
	{\color{myblue}\sffamily\small\bfseries}}
\renewcommand{\subsection}{\@startsection{subsection}{1}{0mm}%
	{.2ex}%
	{.2ex}%x
	{\sffamily\bfseries}}

\makeatother
\setlength{\parindent}{0pt}

%колонтитулы
\pagestyle{fancy}
\fancyhf{}
\setlength{\headheight}{40pt}
\setlength{\headsep}{4pt}
\renewcommand{\headrulewidth}{1pt}
\fancyhead[L]{\textcopyright~\colontitulAutors}
\fancyhead[C]{Программа минимум по курсу <<\colontitulEducationalSubject>> \colontitulYear г}
\fancyhead[R]{Преподаватель:~\colontitulTeacher}

\newcommand{\rot}{\operatorname{rot}}

\begin{document}
	\small
	\begin{multicols*}{2}

		\section{Плоская монохроматическая волна}
		Волна — изменение состояния среды, распространяющееся в данной среде и переносящее с собой энергию. С понятием волны тесно связано понятие физического поля. Поле характеризуется некоторой функцией, определенной в заданной области пространства и времени. Изменение в пространстве и времени большинства полей представляют собой волновой процесс\\
		Монохроматической волной yазывается волна, в которой поле зависит от времени $t$\\
		$U(\vec{r}, t) = A cos(\omega t - \vec{k} \vec{r} + \varphi )$, где $A$ - действительная амплитуда, $\omega$ - циклическая частота, $\varphi $ - начальная фаза, $\vec{k}$ - заданный волновой вектор $(\vec{k} = k_x \vec{e}_x + k_y \vec{e}_y + k_z \vec{e}_z)$, $\theta = (\omega t - \vec{k} \vec{r} + \varphi )$ - полная фаза поля

		\section{Волновое уравнение}
		$\bigtriangleup U - \dfrac{1}{c^2} \dfrac{\partial^2 \vec{U}}{\partial t^2} = 0$ - волновое уравнение без поглощения\\
		$\bigtriangleup U - \beta\dfrac{\partial\vec{U}}{\partial t} -\dfrac{1}{c^2} \dfrac{\partial^2 \vec{U}}{\partial t^2} = 0$ - волновое уравнение c поглощением\\
		Описывает распространение волн различной природы в среде без диссипации.
		$U$ - компонента электрического поля / магнитоного поля / скорость / потенциал, $c$ - имеет смысл фазовой скорости волны, $\beta$ - коэффициент диссипации (учитывает, например, потери в вязкой среди или на нагрев)\\
		Решение - в виде плоской монохроматической волны $U = U_0 e^{(i\omega t - i k \vec{r})}$, если выполнено $\dfrac{\omega^2}{k^2} = c^2$

		\section{Фазовая и групповая скорости}
		$\vec{V}_\text{ф} = \dfrac{\omega}{k^2} \vec{k} = \dfrac{\omega}{k}$ - фазовая скорость (скорость перемещения поверхности постоянной фазы)\\
		$\vec{V}_\text{гр} = \dfrac{\partial\omega}{\partial \vec{k}}\bigg|_{\vec{k_0}}$ - групповая скорость в точке $\vec{k_0}$ (скорость расширения огибающей квазимонохроматического волнового пакета); $\vec{k_0}$ - несущий волновой вектор - максимум спектра квазимонохроматического сигнала\\
		$\vec{V}_\text{ф}$ - скорость движения фронта постоянной фазы, $\vec{V}_\text{гр}$ - скорость движения огибающей

		\section{Уравнение непрерывности и уравнение Эйлера}
		$\dfrac{\partial \rho}{\partial t} + div (\rho \vec{V}) = 0$ - уравнение непрерывности (выражает закон сохранения массы)\\
		% \quad $\dfrac{\partial \rho}{\partial t} + \vec{V} \nabla \rho + \rho div (\vec{V}) = 0$ \quad $\dfrac{\partial \rho}{\partial t} + \rho div (\vec{V}) = 0$
		$\vec{V} (\vec{r}, t)$ - поле скоростей среды, $\mathbf{V} = \frac{1}{\rho}$ - объем на единицу массы, $\left[\rho\right] = \left[\frac{\text{кг}}{\text{м}^3}\right]$\\
		% $\rho \left(\dfrac{\partial \vec{V}}{\partial t} +  (\vec{V} \nabla)  \vec{V}\right)  =  \vec{f} - \nabla p$ - уравн. Эйлера (движение идеал. жидкости в поле внешней силы)\\
		% $\rho$ - плотность жидкости, $p$ - давление, $ \vec{V}$ - вектор скорости, $\vec{f}$ - плотность объемной силы
		$\rho \left(\dfrac{\partial \vec{V}}{\partial t} +  (\vec{V} \nabla)  \vec{V}\right)  = - \nabla p$ - уравн. Эйлера (движение идеал. жидкости в поле внешней силы)\\
		$\rho$ - плотность жидкости, $p$ - давление, $ \vec{V}$ - вектор скорости

		\section{Скорость звука. Вектор Умнова. Плотность энергии в звуковой волне}
		$\sqrt{\dfrac{\gamma k T_0}{m}} = \sqrt{\dfrac{dp}{d\rho}}\bigg|_{\rho_0} = C_s = \sqrt{\dfrac{\gamma R T_0}{M}}$ - адиабатическая скорость звука ($V_\text{ф}$ для звуковой волны)\\
		$\gamma = \dfrac{C_p}{C_V}$ - показатель адиобаты для идеального газа, $T_0$ - равновесное значение температуры, $M$ - молярная масса, $R$ - универсальная газовая постоянная $\left(8.31 \left[\dfrac{\text{Дж}}{\text{моль}\cdot\text{K}}\right]\right)$, $k$ - постоянная Больцмана $\left(1.38\cdot10^{-23} \left[{\text{Дж}} \cdot \text{K}\right]\right)$\\
		$W = \dfrac{\rho_0 V^2}{2} + \dfrac{p_1^2}{2\rho_0 С_s^2}$ - плотность энергии звуковых волн в единице объема \quad СИ: $\left[\dfrac{\text{Дж}^2}{\text{м}^3}\right]$\\
		$\rho_0$ - равновесное значение плотности, $p_1$ - добавочное значение давления: $p = p_0 + p_1$, $\vec{V}$ - скорость распространения возмущения\\
		$\Pi = p_1 \vec{V}$ - плотность потока энергии (вектор Умнова) \quad СИ: $\left[\dfrac{\text{Дж}}{\text{c}\cdot \text{м}^2}\right]$ = $\left[\dfrac{\text{Вт}}{\text{м}^2}\right]$\\
		$\Pi$ - количество энергии, переносимое акустической волной через единичную площадку, перепендикулярную направлению переноса энергии ($\bot \vec{k}$ или $\bot \vec{V}$) в единицу времени (закон сохранения энергии в дифференциальном виде). Направление вектора Умнова - вдоль переноса энергии\\
		Абсолютная величина $p$ равна количеству энергии, переносимому за единицу времени через единичную площадку, перпендикулярную направлению потока энергии.

		\section{Уравнение Ламэ}
		$\rho_0 \dfrac{\partial^2 \vec{U}}{\partial t^2} = (\lambda + \mu) \nabla div \vec{U} + \mu \bigtriangleup  \vec{U}$ - уравнение движения физически бесконечно малого объема изотропного (движение в любых направлениях) упругого тела при малых деформациях\\
		$\rho_0$ - плотность до деформации, $\mu$ - модуль сдвига, $\lambda = K - \frac{2}{3}\mu$ - коэффициент Ламэ, $K$ - модуль всестороннего сжатия, $\vec{U}(\vec{r}, t)$ - вектор смещения элемента сплошной среды при деформации\\
		$\mu$ и $K$ - переобозначения модулей упругости Юнга и Пуассона

		\section{Уравнения Максвелла в дифференциальной и интегральной формах}
        \begin{tabular}{c|lcl|}
            {} & {Дифференциальная форма} & {} & {Интегральная форма} \\
            {1} & {$\rot\vec{E}=-\dfrac1c\dfrac{\partial{\vec{B}}}{\partial{t}}$} & {$\Rightarrow$} & { $\oint\limits_L\vec{E}\dd\vec{L}=-\dfrac1c\dfrac\dd{\dd{t}}\int\limits_S\vec{B}\dd{\vec{S}}$} \\

            {2} & {$\rot\vec{H} = \dfrac{4\pi}{c}\vec{j} + \dfrac1c\dfrac{\partial{\vec{D}}}{\partial{t}}$} & {$\Rightarrow$} & {$\oint\limits_L\vec{H}\dd\vec{L} = \dfrac{4\pi}{c}I+\dfrac1c\dfrac\dd{\dd{t}}\int\limits_S\vec{D}\dd{\vec{S}}$} \\

            {3} & {$\Div\vec{D}=4\pi \rho$} & {$\Rightarrow$} & {$\oint\limits_S\vec{D}\dd{\vec{s}}=4\pi Q$} \\

            {4} & {$\Div\vec{B}=0$} & {$\Rightarrow$} & {$\oint\limits_S\vec{B}\dd{\vec{s}}=0$}
         \end{tabular}
         \begin{enumerate}
             \item {Вихревое электрическое поле поражается переменным магнитным полем.}
             \item {Вихревое магнитное поле порождается токами проводимости и переменным электрическим полем.}
             \item {Потенциальное электрическое поле порождается электрическими зарядами.}
             \item {Магнитное поле имеет чисто вихревой характер и не имеет сосредоточенных зарядов как источников поля.}
         \end{enumerate}

		\section{Граничные условия для векторов ЭМ поля}
        Для нормали из среды 1 в среду 2: \\
        \begin{tabular}{l|l}
            $\left[\vec{n}_{12}\times(\vec E_1-\vec E_2)\right]=0$ &
            $\left(\vec{n}_{12}\cdot(\vec B_1-\vec B_2)\right)=0$ \\
            $\left[\vec{n}_{12}\times(\vec H_1-\vec H_2)\right]=\dfrac{4\pi}{c}\vec j_{\text{пов}}$ &
            $\left(\vec{n}_{12}\cdot(\vec D_1-\vec D_2)\right)=4\pi\vec \rho_{\text{пов}}$
        \end{tabular}

		\section{Вектор Пойнтинга. Плотность энергии ЭМ поля в вакууме}
		$\dfrac{\partial W}{\partial t} + div \vec{S} = - (\vec{j}  \vec{E})$ - теорема Пойнтинга\\
		$W = \dfrac{1}{8\pi} (\mathcal{E}E^2 + \mu H^2)$ - плотность энергии ЭМ поля в вакууме \quad СГС: $\left[\dfrac{\text{эрг}}{\text{см}^{-3}}\right]$\textbf{?check?} \quad СИ: $\left[\dfrac{\text{Дж}}{\text{м}^3}\right]$\\
		$S = \dfrac{c}{4\pi} \left[\vec{E}\times \vec{H}\right]$ - плотность потока энергии \quad СГС: $\left[\dfrac{\text{эрг}}{\text{c}\cdot\text{см}^2}\right]$ \quad СИ: $\left[\dfrac{\text{Дж}}{\text{c}\cdot \text{м}^2}\right]$ = $\left[\dfrac{\text{Вт}}{\text{м}^2}\right]$\\
		$\left\lvert S\right\rvert$ - энергия, переносимая ЭМ волной через единичную площадку $(\bot S)$ в единицу времени\\
		??? проверить + физ смысл

		\section{Основные параметры плазмы (плазменная частота и дебаевский радиус)}
		$r_{De} = \sqrt{\dfrac{k T_e T_i}{4\pi N e^2(T_e + T_i)}} = \sqrt{\dfrac{k T}{4\pi N e^2}}$ - расстояние, за которое волна спадет в $e$ раз при прохождении через плазму / расстояние, которое проходит $\bar{e}$ в плазме за время, порядка $\tau_p = \frac{2\pi}{\omega_p}$\\
		СИ: $\left[{\text{K}}\cdot {\text{Дж}}\right]$
		$T_e$ - температура электронного газа, $T_i$ - температура ионного газа, $N$, $e$ и $m$ - концетрация электронов а также их заряд и масса, $k$ - постоянная Больцмана\\
		$k = \dfrac{R}{N_a}, N_a = \dfrac{m}{M}$ ???????????????\\
		$\omega_p = \dfrac{4\pi e^2 N}{m}$ - плазменная частота, СИ: $\left[\dfrac{\text{рад}}{\text{c}}\right]$ ???\\
		Это частота собственных продольных колебаний пространственного заряда в однородной плазме в отсутствие магнитного поля

		\section{Комплексная диэлектрическая проницаемость холодной изотропной плазмы}
		Диэлектрическая проницаемость показывает, во сколько раз сила взаимодействия двух электрических зарядов в конкретной среде меньше, чем в вакууме, для которого она равна $1$\\
		$\mathcal{E} (\omega) = 1 - \dfrac{\omega_{pe}^2}{\omega(\omega - i\nu_e)} - \chi $, где $\chi = \dfrac{\omega_{pi}^2}{\omega(\omega - i\nu_i)}$ - ионная составляющая, которой можно пренебречь, $\nu_e$ - частота соударений электронов\\
		Вводятся абсолютная $(\mathcal{E}_a)$ и относительная $(\mathcal{E}_r)$ проницаемости. Величина $\mathcal{E} _{r}$ безразмерна, а ${\displaystyle \mathcal{E} _{a}}$ по размерности совпадает с электрической постоянной $\mathcal{E}_{0}$ - СИ: $\left[\dfrac{\text{фарад}}{\text{м}}\right]$\\
		Эта величина связывет напряженность и индукцию поля: $D = \mathcal{E} E $


	\end{multicols*}
\end{document}

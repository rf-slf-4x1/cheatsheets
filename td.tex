% Подписи колонтитула
\newcommand{\colontitulAutors}{astronom\_v\_cube,~edombek}
\newcommand{\colontitulYear}{2023}
\newcommand{\colontitulEducationalSubject}{Термодинамика и статистическая физика}
\newcommand{\colontitulTeacher}{Гавриленко В.Г.}

%Настройки шаблона
\documentclass[10pt,landscape,a4paper]{article}
\usepackage[utf8]{inputenc}
\usepackage[english, russian]{babel}
\usepackage[T1,T2A]{fontenc}  
\usepackage{upgreek} % прямые греческие ради русской традиции
\usepackage{tikz}
\usetikzlibrary{shapes,positioning,arrows,fit,calc,graphs,graphs.standard}
%\usepackage[nosf]{kpfonts}
%\usepackage[t1]{sourcesanspro}
\usepackage{multicol}
\usepackage{wrapfig}
\usepackage[top=6mm,bottom=8mm,left=4mm,right=4mm]{geometry}
\usepackage[framemethod=tikz]{mdframed}
\usepackage{microtype}
\usepackage{pdfpages}
\usepackage{amsthm,amsmath,amscd}   % Математические дополнения от AMS
\usepackage{amsfonts,amssymb}       % Математические дополнения от AMS
\usepackage{mathtools}              % Добавляет окружение multlined
\usepackage{xfrac}                  % Красивые дроби
\usepackage{physics}

\usepackage{fancyhdr} % колонтитулы

%некоторые математические команды
\newcommand{\Div}{\operatorname{div}}
\newcommand{\Grad}{\operatorname{grad}}

\let\bar\overline

\definecolor{myblue}{cmyk}{1,.72,0,.38}

\def\firstcircle{(0,0) circle (1.5cm)}
\def\secondcircle{(0:2cm) circle (1.5cm)}

\colorlet{circle edge}{myblue}
\colorlet{circle area}{myblue!5}

\tikzset{filled/.style={fill=circle area, draw=circle edge, thick},
	outline/.style={draw=circle edge, thick}}

\pgfdeclarelayer{background}
\pgfsetlayers{background,main}

%\everymath\expandafter{\the\everymath \color{myblue}}
\everydisplay\expandafter{\the\everydisplay \color{myblue}}

\renewcommand{\baselinestretch}{.8}
\pagestyle{empty}

\global\mdfdefinestyle{header}{%
	linecolor=gray,linewidth=1pt,%
	leftmargin=0mm,rightmargin=0mm,skipbelow=0mm,skipabove=0mm,
}

\makeatletter % Author: ttps://tex.stackexchange.com/questions/218587/how-to-set-one-header-for-each-page-using-multicols
\renewcommand{\section}{\@startsection{section}{1}{0mm}%
	{.2ex}%
	{.2ex}%x
	{\color{myblue}\sffamily\small\bfseries}}
\renewcommand{\subsection}{\@startsection{subsection}{1}{0mm}%
	{.2ex}%
	{.2ex}%x
	{\sffamily\bfseries}}

\makeatother
\setlength{\parindent}{0pt}

%колонтитулы
\pagestyle{fancy}
\fancyhf{}
\setlength{\headheight}{40pt}
\setlength{\headsep}{4pt}
\renewcommand{\headrulewidth}{1pt}
\fancyhead[L]{\textcopyright~\colontitulAutors}
\fancyhead[C]{Программа минимум по курсу <<\colontitulEducationalSubject>> \colontitulYear г}
\fancyhead[R]{Преподаватель:~\colontitulTeacher}

\begin{document}
	\small
	\begin{multicols*}{2}

		\section{Каноническое распределение Гиббса.}
		Распределение вероятностей различных возможных состояний некоторой квазизамкнутой(некоторая часть замкнутой макроскопической системы) подсистемы. Подсистема называется квазизамкнутой, если её собственная энергия в среднем велика по сравнению с энергией её взаимодействия с остальными частями замкнутой системы(называемыми термостатами).
		$H_1(q_1,p_1)+H_2(q_2,p_2)+H_{\text{вз}}(q_1,p_1,q_2,p_2)=E=\const$ (изолированная система)\\
		$H_{\text{вз}} \ll H_1,h_2$ \\
		$\rho(p,q)=\frac{1}{z_{\text{кл}}}\exp\left[-\dfrac{H(p,q)}{\theta}\right]$ - \textbf{Каноническое распределение Гиббса}\\
		$\theta$ - модуль распределения Гиббса, из условий нормировки($\int\rho(p,q)=1$):\\
		$z_{\text{кл}}=\int\limits_{V_{21}}e^{-\frac{H(p,q)}{\theta}}\dfrac{dpdq}{dW_{21}}$ - статистический интеграл.
		
		\section{Статистическое и термодинамическое определение энтропии.}
		\begin{itemize}
			\item Статистическое определение энтропии:\\
			$S=k\ln(\Delta\Gamma)$\\
			$\Delta\Gamma \geq 1 \Rightarrow S = 0$
			\item Термодинамическое распределение энтропии:\\
			$dS=\dfrac{dQ}{T}$ - для квазистатического процесса
		\end{itemize}
		Где $k=1.38*10^{-16}$ эрг/град - постоянная Больцмана;
		$\Delta\Gamma$ - Статистический вес состояния (число микросостояний, которые возможны в имеющимся макроскопическом состоянии)

		\section{Температура. Термодинамический и статистический смысл.}
		Температура $\tau$ - это функция состояния, принимающая одинаковые значения для всех систем (тел), находящихся в термодинамическом равновесии, и определяемая внешними параметрами и энергией системы. Температура любой системы должна быть монотонной функцией её энергии, причем для всех частей системы эти функции должны быть либо монотонно убывающими, быть либо монотонно возрастающими.\\
		\textbf{???}

		\section{Первый принцип термодинамики.}
		Физически первое начало термодинамики - это уравнение баланса энергии\\
		$dU = \text{\dj}Q^* + \text{\dj}A^*$, или $dU = \text{\dj}Q^* - \text{\dj}A$\\
		$U$ - внутренняя энергия макроскопической системы. Внутренняя энергия - функция состояния, разность значений которой в двух состояниях $U_2 - U_1$ равна работе, которую надо совершить над системой при теплоизолированном переходе из первого состояния во второе: $A^*_{12} = U_2 - U_1 = Q^*_{12}$\\
		Звездочка "$*$" означает воздействие на систему извне. В отсутствие теплоизоляции работа над системой и количество полученного тепла при переходе из одного состояния в другое зависят от способа перехода. Поэтому элементарная работа над системой $\text{\dj}A^*$ и элементарное количество полученного тепла $\text{\dj}Q^*$ не являются полными дифференциалами и часто обозначаются с использованием символа $\text{\dj}$

		\section{Теплоемкость.}
		Теплоемкость - это количество тепла, которое нужно передать системе для увеличения её температуры на одну единицу измерения, например, один градус. Из первого принципа термодинамики следует, что теплоёмкость зависит не только от того, насколько изменилась внутренняя энергия, но и от того, какую работу совершила система, то есть от процесса, при котором происходит нагревание.

		\section{Второй принцип термодинамики.}
		Существуют различные варианты его формулировки, вытекающие один из другого.\\
		1. Нельзя построить циклически работающий двигатель, переносящий тепло от холодного тела к горячему, без превращения механической энергии во внутреннюю (невозможно создать идеальную холодильную машину). Это эквивалентно утверждению о том, что невозможна самопроизвольная передача тепла от холодного тела к горячему.\\
		2. Нельзя построить циклически работающий двигатель, единственным действием которого является полное превращение тепла от резервуара в механическую работу (невозможно создать вечный двигатель второго рода).\\
		3. Существуют адиабатически (теплоизоизолированно) недостижимые состояния.\\
		Все эти формулировки говорят о существовании для макроскопических систем необратимых процессов, которые могут протекать только в одном направлении. Действительно, вполне возможна самопроизвольная передача тепла от горячего тела к холодному, но не наоборот. Вполне возможно полное превращение механической работы в тепло (например, за счёт трения), но обратный процесс неосуществим.

		\section{Цикл Карно.}

		\section{Характеристические функции $F, \Phi, I$.}

		\section{Фазовые переходы первого рода.}

		\section{Распределение Максвелла-Больцмана.}
		Распределение, характеризующее вероятность того, что молекула имеет длинный импульс и находится в данном элементе объёма.\\
		$dw=\dfrac 1 {(2\pi mkT)^{3/2}}\exp\left[-\dfrac{\vec {p}^2}{2mkT}\right]d\vec p * \dfrac{\exp\left[-\frac{U(V)}{kT}\right]dV}{\int\exp\left[-\frac{U(V)}{kT}\right]dV}$ \\
		Где дробь - Распределение Больцмана, а остальная часть - распределение Максвелла.

		\section{Формула Планка для равновесного излучения.}

		\section{Третье начало термодинамики.}
		При стремлении температуры к абсолютному  нулю энтропия всякой равновесной системы стремится к одному и тому же для всех систем конечному значению, которое можно положить равным нулю.\\
		$\lim\limits_{T\to 0} [ S(T, x_2) - S(T, x_1)] =0$, или $\lim\limits_{T\to 0} \left(\dfrac{\partial S}{\partial x}\right)\vert _T = 0$\\
		$x_1, x_2$ - набор макроскопических (внешних) параметров, от которых зависит состояние системы\\
		$\lim\limits_{x \to \ell }f(x)=L$

		\section{Распределение Бозе.}

		\section{Распределение Ферми.}

		\section{Критерий невырожденности идеального газа.}
		Невырожденность - это отсутствие квантовых связей\\
		Критерий невырожденности: $\dfrac{(2\pi \hbar)^3 N}{V(2\pi mkT)^{\frac{2}{3}}} \ll 1$ эквивалентен условию $T\gg T_{\text{выр}} = \dfrac{\hbar^2}{mk} \left(\dfrac{N}{V}\right)^{\frac{2}{3}}$

		\section{Энергия Ферми.}
		${\mathcal{E}}_{\text{max}} = \dfrac{p^2_max}{2m} = \left(\dfrac{3N}{8\pi V}\right)^{\frac{2}{3}}\dfrac{h^2}{2m}$ - граничная энергия (вырождения) Ферми газа
	\end{multicols*}
\end{document}

% Подписи колонтитула
\newcommand{\colontitulAutors}{edombek, astronom\_v\_cube и другие}
\newcommand{\colontitulYear}{2022}
\newcommand{\colontitulEducationalSubject}{МЕХАНИКА СПЛОШНЫХ СРЕД}
\newcommand{\colontitulTeacher}{С.~Н.~Гурбатов}

%Настройки шаблона
\documentclass[10pt,landscape,a4paper]{article}
\usepackage[utf8]{inputenc}
\usepackage[english, russian]{babel}
\usepackage[T1,T2A]{fontenc}  
\usepackage{upgreek} % прямые греческие ради русской традиции
\usepackage{tikz}
\usetikzlibrary{shapes,positioning,arrows,fit,calc,graphs,graphs.standard}
%\usepackage[nosf]{kpfonts}
%\usepackage[t1]{sourcesanspro}
\usepackage{multicol}
\usepackage{wrapfig}
\usepackage[top=6mm,bottom=8mm,left=4mm,right=4mm]{geometry}
\usepackage[framemethod=tikz]{mdframed}
\usepackage{microtype}
\usepackage{pdfpages}
\usepackage{amsthm,amsmath,amscd}   % Математические дополнения от AMS
\usepackage{amsfonts,amssymb}       % Математические дополнения от AMS
\usepackage{mathtools}              % Добавляет окружение multlined
\usepackage{xfrac}                  % Красивые дроби
\usepackage{physics}

\usepackage{fancyhdr} % колонтитулы

%некоторые математические команды
\newcommand{\Div}{\operatorname{div}}
\newcommand{\Grad}{\operatorname{grad}}

\let\bar\overline

\definecolor{myblue}{cmyk}{1,.72,0,.38}

\def\firstcircle{(0,0) circle (1.5cm)}
\def\secondcircle{(0:2cm) circle (1.5cm)}

\colorlet{circle edge}{myblue}
\colorlet{circle area}{myblue!5}

\tikzset{filled/.style={fill=circle area, draw=circle edge, thick},
	outline/.style={draw=circle edge, thick}}

\pgfdeclarelayer{background}
\pgfsetlayers{background,main}

%\everymath\expandafter{\the\everymath \color{myblue}}
\everydisplay\expandafter{\the\everydisplay \color{myblue}}

\renewcommand{\baselinestretch}{.8}
\pagestyle{empty}

\global\mdfdefinestyle{header}{%
	linecolor=gray,linewidth=1pt,%
	leftmargin=0mm,rightmargin=0mm,skipbelow=0mm,skipabove=0mm,
}

\makeatletter % Author: ttps://tex.stackexchange.com/questions/218587/how-to-set-one-header-for-each-page-using-multicols
\renewcommand{\section}{\@startsection{section}{1}{0mm}%
	{.2ex}%
	{.2ex}%x
	{\color{myblue}\sffamily\small\bfseries}}
\renewcommand{\subsection}{\@startsection{subsection}{1}{0mm}%
	{.2ex}%
	{.2ex}%x
	{\sffamily\bfseries}}

\makeatother
\setlength{\parindent}{0pt}

%колонтитулы
\pagestyle{fancy}
\fancyhf{}
\setlength{\headheight}{40pt}
\setlength{\headsep}{4pt}
\renewcommand{\headrulewidth}{1pt}
\fancyhead[L]{\textcopyright~\colontitulAutors}
\fancyhead[C]{Программа минимум по курсу <<\colontitulEducationalSubject>> \colontitulYear г}
\fancyhead[R]{Преподаватель:~\colontitulTeacher}

\newcommand{\sumk}{\sum\limits_{k=1}^3}

\begin{document}
	\small
	\begin{multicols*}{2}
		\section{Понятие субстанциальной и локальной производных.}
		$\dfrac{d}{dt}=\dfrac{\partial}{\partial t}+(\vec{v}\nabla)$ - субстанциальная \\
		$\dfrac{\partial}{\partial t}$~-~локальная
		
		\section{Уравнение неразрывности для сжимаемой и несжимаемой жидкости.}
		$\dfrac{d\rho}{dt}+\rho \Div(\vec{v})=0$, \\
		$\dfrac{d\rho}{dt}=0$ -  для несжимаемой ($\Div(\vec{v})=0$)
		
		\section{Уравнение Эйлера в векторной форме и в проекциях на оси в декартовой системе координат.}
		Уравнение Эйлера описывает движение идеальной жидкости\\
		$\dfrac{d\vec{v}}{dt}=-\dfrac{\nabla p}{\rho}+\vec{f}$ \\
		$\dfrac{\partial v_i}{\partial t}+\sumk v_k\dfrac{\partial v_i}{\partial x_k}=-\dfrac{1}{\rho}\dfrac{\partial p}{\partial x_i}+f_i$
		
		\section{Закон сохранения энергии идеальной жидкости. Поток энергии.}
		$\int\limits_V\left[\dfrac{\partial}{\partial t}(\dfrac{\rho v^2}{2}+\rho\varepsilon)+\Div(\dfrac{\rho v^2}{2}+W)\vec{v}\right]dV=0$, где\\
		$W=\rho\varepsilon+p$ - энтальпия, $\varepsilon$ - плотность энергии на единицу массы \\
		или в дифференциальной форме \\
		$\dfrac{\partial E}{\partial t}+div\vec{N}=0$, где \\
		$E=\dfrac{\rho v^2}{2}+\rho\varepsilon$ - плотность энергии \\
		$\vec{N}=\left[\dfrac{\rho v^2}{2}+\rho\varepsilon+p\right]\vec{v}$ - вектор плотности потока энергии
		
		\section{Закон сохранения импульса идеальной жидкости. Тензор плотности потока импульса и его представление в декартовой системе координат.}
		$ \dfrac{\partial}{\partial t}\int\limits_V \rho\vec{v}dV=-\oint\limits_S \left[p\vec{n}+\rho\vec{v}(\vec{v}\vec{n})\right]d\sigma$ \\
		$ \dfrac{\partial}{\partial t}(\rho v_i)=-\sumk\dfrac{\partial \Pi_{ik}}{\partial x_k}+\rho f_i $ \\
		$ \Pi_{ik} = p\delta_{ik}+\rho v_iv_k$ - тензор ППИ
		
		\section{Уравнение гидростатики.}
		$\Grad p=\rho\vec{f}, \quad p=p(\rho)$
		
		\section{Частота Брента-Вяйсяля.}
		$N=\sqrt{\dfrac{g}{\rho}\dfrac{d\rho}{dz}}$\\
		Если $N^2<0$, то неустойчивость жидкости (тело всплывает или тонет). Если $N^2>0$, то жидкость устойчива (тело не двигается).
		
		\section{Теорема Бернулли для потенциальных и не потенциальных, стационарных и нестационарных течений.} 
		$\dfrac{v^2}{2}+\dfrac{p}{\rho}-gz=const$ - стационарное безвихревое ($const$ во всём объёме)\\
		$\dfrac{v^2}{2}+W-gz=const$ - стационарное вихревое ($const$ на линии тока)\\
		$\dfrac{\partial \varphi}{\partial t}+\dfrac{v^2}{2}+\dfrac{p}{\rho}-gz=const$ - нестационарное безвихревое \\
		
		\section{Теорема Томсона.}
		Циркуляция скорости($\Gamma$) вдоль замкнутого контура, перемещающегося в идеальной жидкости, остается постоянной. \\
		$\Gamma = \oint\limits_L\vec{v}d\vec{r}=const$
		
		\section{Потенциальные течения идеальной несжимаемой жидкости. Основные уравнения, граничные условия.}
		$\Delta \varphi=0,\quad \vec{v}=\Grad(\phi)$ \\
		Граничное условие не протекания: \\
		Нормальная компонента скорости на границе с телом равна нулю.\\
		$\vec{v}\vec{n}|_s=\dfrac{\partial\varphi}{\partial n}=\vec{v_0}\vec{n}$ \\
		Граничное условие на бесконечности: \\
		используют значение потенциала на бесконечности.\\
		%\textbf{???} $\vec{v}\vec{n}|_\infty =\dfrac{\partial\varphi}{\partial n}=0$ \textbf{???}
		
		\section{Парадокс Д’Аламбера-Эйлера.}
		\textbf{Ф1.} При обтекании тела с гладкой поверхностью идеальной
		несжимаемой жидкостью сила лобового сопротивления, действующая на тело
		со стороны потока, равна нулю. \\
		\textbf{Ф2.} Для тела, движущегося равномерно в идеальной несжимаемой жидкости постоянной плотности без границ, сила сопротивления равна
		нулю.\\
		$\vec{F}=-\oint\limits_sp_s\vec{n}dS=0$
		
		\section{Понятие присоединенной массы. Присоединенная масса сферы и единицы длины бесконечного кругового цилиндра.}
		Присоединенная масса - это масса, которая добавляется к массе тела, движущегося неравномерно в жидкой среде для учета воздействия среды на это тело. 
		($M=F_\text{сопр}/a$) \\
		$M_\text{сферы}=\dfrac{2}{3}\rho\pi R^3$ (равна половине массы вытесненной жидкости)\\
		$M_\text{цилиндра}=\rho(\pi R^2*1)$ (равна массе вытесненной жидкости)
		
		\section{Функция тока и ее свойства.}
		Для плоского потенциального течения несжимаемой идеальной жидкости: \\
		$\psi = \psi (x, y, t); \quad v_{x}=\dfrac{\partial\psi}{\partial y}; \quad v_{y}=-\dfrac{\partial\psi}{\partial x}$ \\
		$d\psi = \dfrac{\partial\psi}{\partial x}dx + \dfrac{\partial\psi}{\partial y}dy = -v_{y}dx + v_{x}dy$\\
		На линии тока $\psi = const$. Линии тока ортогональны линиям уровня ($(\nabla\psi, \nabla\varphi)=0$). Функция тока - является гармонической функцией, удовлетворяющей уравнению Лапласа $\Delta \psi = 0$. 
		
		\section{Комплексный потенциал.}
		$F(z)=\phi+i\Psi$ (действительная часть - потенциал, мнимая – функция тока)\\
		Любую аналитическую функцию комплексного переменного можно поставить в соответствие с неким плоским потенциальным течением идеальной несжимаемой жидкости.
		
		\section{Линии тока и эквипотенциальные линии.}
		\textbf{Линия тока} - это линия, касательные к которой в данный момент времени и
		в каждой точке совпадают с вектором скорости  $\vec{v}$ \\
		$\Psi = const$ - линии тока (постоянная функция тока) \\
		$\varphi = const$ - эквипотенциальные линии (постоянный потенциал)
		
		\section{Формула Жуковского.}
		$F_y=-\int pn_ydl=\rho\Gamma v_0$\\
		Сила пропорциональна плотности, скорости и параметру, характеризующему вихрь.
		
		\section{Точечные вихри и их взаимодействия.}
		Устремляем сечение нашей вихревой трубки к нулю, а частоту к бесконечности - получаем точечный вихрь. Скорость точечного i-ого вихря равна скорости жидкости в данной точке, создаваемой всеми остальными вихрями.\\
		$\dfrac{d\vec{r}_i }{dt} = \sum\limits_{k\neq i}{\vec{v_k}(\vec{r_i})}$
		
		\section{Поверхностные гравитационные волны (длинные, короткие, гравитационно-капиллярные) и их основные свойства (траектории движения частиц, дисперсионные уравнения, фазовые и групповые скорости).}
		Траектории описываются уравнением эллипса: \\
		$\dfrac{\xi^2}{a_\xi^2}+\dfrac{\eta^2}{a_\eta^2} = 1, \quad a_\xi = \dfrac{a\sh{k(z+H)}}{\sh{kH}}	, \quad a_\eta = \dfrac{a\ch{k(z+H)}}{\sh{kH}}$ \\
		$\xi$ и $\eta$ - смещения частицы по вертикали и горизонтали соответственно\\
		$ \xi = \dfrac{a}{\sh{kH}}\sh{k(z+H)}\cos{(kx-\omega t)} $\\
		$ \eta = -\dfrac{a}{\sh{kH}}\ch{k(z+H)}\sin{(kx-\omega t)} $\\
		Дисперсионное уравнение: \\
		$\omega^2=gk\th{kH} = gk \frac{e^{kH}-e^{-kH}}{e^{kH}+e^{-kH}} = gk\frac{\sh kH}{\ch x}$ \\
		В случае волн на мелкой воде:\\
		$\omega^2=gk^2H, \quad \omega=\pm k\sqrt{gH}$,	$v_\text{ф}=\frac{\omega}{k},	v_\text{гр}=\dv{\omega}{k}$\\
		В случае волн на глубокой воде:\\
		$\omega=\pm\sqrt{gk}, \quad v_\text{ф}=\sqrt{\frac{g}{k}},	\qquad v_\text{гр}=\frac{g}{2\sqrt{gk}}=\frac12 v_\text{f}$\\
		В случае гравитационно-капиллярных волн:\\
		$\omega^2 = (gk+\gamma k^3)\th{kH},		\gamma = \frac{\alpha}{\rho}$\\
		$ v_\text{ф}^2 = {\omega^2}{k^2} = {g}{k}+\gamma k$ \\
		$k_{*} = \sqrt{{g}{k}}$ - минимум $v_\text{ф}$\\
		$v_\text{гр} = \dfrac{d\omega}{dk} \quad \Rightarrow \quad$
		$v_\text{гр} = \dfrac{v_\text{ф}}{2}\dfrac{k_{*}^2+3k^2}{k_{*}^2+k^2} $ \\
		Если $k \gg k_*$, это капиллярные волны. 
		Если ${H} \ll k \ll k_*$, то это гравитационные короткие волны (дно ещё не чувствуется).
		Если же $k \ll {H}$, то это длинные гравитационные волны.
		
		\section{Уравнение Навье-Стокса для несжимаемой вязкой жидкости в векторной форме и в проекциях на оси в декартовой системе координат.}
		Запись через кинематическую вязкость $\nu=\eta/\rho$:\\
		$\dfrac{\partial \vec{v}}{\partial t}+(\vec{v} \nabla)\vec{v} = -\dfrac{\nabla p}{\rho}+\nu \Delta \vec{v}$\\
		$\dfrac{\partial {v_i}}{\partial t} + \sumk v_k \dfrac{\partial {v_i}}{\partial x_k} = -\dfrac{1}{\rho}\dfrac{\partial p }{\partial x_i} + \nu\sumk\dfrac{\partial^2 {v_i}}{\partial {x^2_k}}$
		
		\section{Тензор вязких напряжений, физический смысл, представление в декартовой системе координат.}
		Общий вид тензора вязких напряжения (при относительном смещении слоёв жидкости, зависимость $\sim\eta$ линейна, жидкость будем считать изотропной): \\
		$\sigma_{ik} = a\qty(\pdv{v_i}{x_k}+\pdv{v_k}{x_i})+c\qty(\pdv{v_i}{x_k}-\pdv{v_k}{x_i})+b\sum \pdv{v_l}{x_l} \delta_{ik}$ \\
		Переобозначим константы $a=\eta$, $b=\xi$. Тогда тензор вязких напряжений перепишется как\\
		$ \sigma_{ik} = \eta \qty(\pdv{v_i}{x_k}+\pdv{v_k}{x_i})+\xi \sum\limits_l \pdv{v_l}{x_l} \delta_{ik} $\\
		Тензор вязких напряжений - это тензор, используемый для моделирования части напряжения в точке внутри некоторого материала, которая может быть отнесена к скорости деформации, т.е. скорости, с которой материал деформируется вокруг этой точки.
		
		\section{Граничные условия для несжимаемой вязкой жидкости на поверхности твердого тела и свободной поверхности.}
		В случае вязкой жидкости скорость жидкости на границе с телом равна скорости тела: \\
		$v_\text{жидк}|_s=v_\text{тела}$\\
		При рассмотрении гидродинамики слоя жидкости на верхней границе: \\
		$ f_i = \sigma_{ik} n_k = \eta\pdv{v_x}{y} = 0 $
		
		\section{Формула Пуазейля для расхода жидкости.}
		$Q=2\pi\int\limits_0^Rv(r)rdr=\dfrac{\pi}{8\eta}\left(\dfrac{\partial p}{\partial z}\right)R^4$
		
		\section{Скин-слой.}
		Поскольку среда вязкая, возмущения передаются наверх, но затухают на характерном масштабе толщины скин-слоя \\
		$\delta=\sqrt{\dfrac{2\nu}{\omega}}$
		
		\section{Числа Рейнольдса, Фруда, Струхаля и их физический смысл.}
		$Re=\dfrac{v_0l}{\nu}=\dfrac{2v_0R}{\nu} = \dfrac{V_\text{ср}H}{\nu}$\\
		$\nu=\dfrac{\eta}{\rho}$ - кинематический коэффициент вязкости \\
		Число Рейнольдса показывает относительное влияние нелинейных эффектов. Если Re мало, то можно пренебречь в уравнении движения вязкой жидкости всем, кроме давления. \\
		$Fr=\dfrac{v_0^2}{gl}$ \\
		Число Фруда описывает отношение кинетической энергии жидкости к потенциальной (энергии гравитационных сил). \\
		$Sh=\dfrac{v_0T}{l}$ \\
		Число Струхаля характеризует стационарность. Если $Sh >> 1$ можно пренебречь нестационарностью.
		
		\section{Формула Стокса.}
		Сила сопротивления, действующая на маленькое тело, движущееся в жидкости.\\
		$F=6\pi\eta Rv_0, ~Re << 1$ \\
		$F=6\pi\eta Rv_0\left(1+\dfrac{3}{16}Re\right)$
		
		\section{Зависимость ширины пограничного слоя от параметров.}
		Пограничный слой - слой, где скорость меняется от нуля до скорости, соответствующей скорости обтекания тела идеальной жидкостью.\\
		Во-первых, чем больше вязкость, тем толще пограничный слой. Кроме того, чем дальше по $x$, тем слой толще. И, наконец, чем больше скорость, тем больше пограничный слой должен быть прижат к пластине.
		
		\section{Уравнения линейной акустики. Волновое уравнение.}
		Уравнение Эйлера, уравнение непрерывности и последнее уравнение - состояния:\\
		$\dfrac{\partial \vec{v}'}{\partial t} = -\dfrac{\nabla p'}{\rho_0}, \quad \dfrac{\partial \rho'}{\partial t} + \rho_0 c^2 \Div \vec{v} = 0, \quad p' = c^2 \rho'$\\
		Здесь $p = p_0 + p', \quad v = v_0 + v', \quad\rho  = \rho_0 + \rho'$\\
		$p_0 >> p', \quad v_0 >> v', \quad \rho_0 >> \rho'$\\
		Величины с индексом 0 равновесная среда, штрихами обозначены добавки, возникающие при распространении звука.\\
		$M = \dfrac{v}{c} = \dfrac{p'}{p_0} = \dfrac{\rho'}{\rho_0}$ - число Маха, $M << 1$\\
		Волновое уравнение:
		$\dfrac{\partial^2 \varphi}{\partial t^2} - c^2 \Delta\varphi = 0$
		
		\section{Монохроматические волны, уравнение Гельмгольца}
		Уравнение Гельмгольца:
		$\Delta \Phi_0+k_0^2 \Phi_0 = 0,\quad k_0=\dfrac{\omega}{c}$\\
		Простейшее решение - плоские волны:
		$\Phi_0 = e^{i\qty(\vec{k}, \vec{r})}$\\
		В случае $\vec{k} = \vec{k}_1 + i\vec{k}_2$ (неоднородная плоская волна):\\
		$\Phi_0 = e^{i\qty(\vec{k}_1, \vec{r})}e^{-\qty(\vec{k}_2,\vec{r})}$.
		Всякую волну можно представить в виде суперпозиции плоских монохроматических волн с различными волновыми векторами и частотами.
		
		\section{Закон сохранения энергии (звуковой волны)}
		$\dfrac{\partial E}{\partial t} + \Div\vec{J}=0$\\
		$\vec{J} = \rho \vec{v}$ - вектор Умова-Пойнтинга - интенсивность звуковой волны, сила звука. \\
		$E$ - полная энергия звуковой волны.
		
	\end{multicols*}
\end{document}